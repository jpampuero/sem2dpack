
\documentclass[letterpaper,11pt,titlepage,final]{report}

\usepackage{epsfig,pifont,calc,color,pstricks}
\usepackage{amsmath,latexsym,oldlfont}
\usepackage{enumerate}
\usepackage{Chicago}
\usepackage{url}

%\def\ImPath{/home/ampuero/WAVES/SEM2DPACK_2.x/doc/FIGURES}
\def\ImPath{../FIGURES}
\def\Version{2.3.8}
\def\MonthYear{August 2012}

%%%%%%%%%%%%%%%%%%%%%%%%%%%%
%% A4 wide
%%%%%%%%%%%%%%%%%%%%%%%%%%%%
 
\usepackage{a4wide}
%\textwidth 16.7cm
%\textheight 24.6cm
%\oddsidemargin -4.5mm
%\evensidemargin -4.5mm
%\topmargin -11.5mm
 
%%%%%%%%%%%%%%%%%%%%%%%%%%%%
%% US letter
%%%%%%%%%%%%%%%%%%%%%%%%%%%%
 
%\textwidth 17cm
%\textheight 23.8cm
%\oddsidemargin -4.5mm
%\evensidemargin -4.5mm
%\topmargin -22.5mm
 
%%%%%%%%%%%%%%%%%%%%%%%%%%%%

%\usepackage{fancyheadings}
\usepackage{fancyhdr}
\pagestyle{fancyplain}
\renewcommand{\chaptermark}[1]{\markboth{#1}{#1}}
\renewcommand{\sectionmark}[1]{\markright{\thesection\ #1}}
\lhead[\fancyplain{}{\bfseries\thepage}]{\fancyplain{}{\bfseries\rightmark}}
\rhead[\fancyplain{}{\bfseries\leftmark}]{\fancyplain{}{\bfseries\thepage}}
\cfoot{}
\renewcommand{\labelitemii}{{\LARGE\bf .}}

% pour voir la bounding box postscript
%\newcommand{\voirbbox}[1]{{\fboxsep 0pt \fbox{#1}}}

% pour ne pas la voir
\newcommand{\voirbbox}[1]{{{#1}}}

%
% define new items (necessite package pifont)
%

\font\pzdr=pzdr at 12pt

\newenvironment{fancyitemize}[1]
   {\begin{itemize}
    \renewcommand{\labelitemi}{{\pzdr #1}}
    \renewcommand{\labelitemii}{{\pzdr #1}}
    \renewcommand{\labelitemiii}{{\pzdr #1}}
   }
   {\end{itemize}}

%
% define new enumerate (necessite package pifont et calc)
%

\newcounter{local}

\newenvironment{enumerateW}
   {
     \begin{enumerate}
     \renewcommand{\labelenumi}{\setcounter{local}{191+\value{enumi}}\ding{\value{local}}}
     \renewcommand{\labelenumii}{\setcounter{local}{191+\value{enumii}}\ding{\value{local}}}
     \renewcommand{\labelenumiii}{\setcounter{local}{191+\value{enumiii}}\ding{\value{local}}}
   }
   {\end{enumerate}}

\newenvironment{enumerateB}
   {
     \begin{enumerate}
     \renewcommand{\labelenumi}{\setcounter{local}{201+\value{enumi}}\ding{\value{local}}}
     \renewcommand{\labelenumii}{\setcounter{local}{201+\value{enumii}}\ding{\value{local}}}
     \renewcommand{\labelenumiii}{\setcounter{local}{201+\value{enumiii}}\ding{\value{local}}}
   }
   {\end{enumerate}}

\newcommand{\relief}[1]{{\it\bfseries #1}}

%%%%%%%%%%%%%%%%%%%%%%%%%%%%%%%%%%%%%%%%%%%%%%%%%%%%%%%%%%%%%%%%%%
% FIGURES:

\newcommand{\Img}[2]{
\begin{minipage}{#2\linewidth}
\centerline{ \epsfig{file=\ImPath/#1,width=\linewidth}}
\end{minipage}
}

\newcommand{\ImgC}[2]{
\centerline{ \epsfig{file=\ImPath/#1,width=#2\linewidth} }
}

\newcommand{\ImgL}[2]{
\begin{minipage}{#2\linewidth}
\centerline{ \epsfig{file=\ImPath/#1,width=\linewidth,angle=-90}}
\end{minipage}
}

\newcommand{\ImgCL}[2]{
\centerline{ \epsfig{file=\ImPath/#1,width=#2\linewidth,angle=-90} }
}

%%%%%%%%%%%%%%%%%%%%%%%%%%%%%%%%%%%%%%%%%%%%%%%%%%%%%%%%%%%%%%%%%%%%%
% EQUATIONS:
\newcommand{\eq}{\begin{equation}}
\newcommand{\en}{\end{equation}}
\newcommand{\eqa}{\begin{eqnarray}}
\newcommand{\ena}{\end{eqnarray}}

%%%%%%%%%%%%%%%%%%%%%%% REFERENCES %%%%%%%%%%%%%%%%%%%%%%%%%%%%%%%%%%%
\newcommand{\Eqref}[1]{Equation~\ref{Eq:#1}}
\newcommand{\figref}[1]{Figure~\ref{Fig:#1}}
\newcommand{\secref}[1]{Section~\ref{Sec:#1}}
\newcommand{\charef}[1]{Chapter~\ref{Cha:#1}}
\newcommand{\appref}[1]{Appendix~\ref{App:#1}}
\newcommand{\tabref}[1]{Table~\ref{Tab:#1}}

%%%%%%%%%%%%%%%%%%%%%%%%%%%%%%%%%%%%%%%%%%%%%%%%%%%%%%%%%%%%%%%%%%%%%
% SHORTER ITEMIZE
\newenvironment{sitemize}
{\begin{list}{--}%
  {\setlength{\topsep}{0pt}%
   \setlength{\itemsep}{0pt}%
   \setlength{\leftmargin}{2em} }}
{\end{list}}

\newenvironment{senumerate}
{\begin{enumerate}
  \setlength{\topsep}{0pt}
  \setlength{\itemsep}{0pt}
  \setlength{\parskip}{0pt}
  \setlength{\parsep}{0pt}}
{\end{enumerate}
}

%%%%%%%%%%%%%%%%%%%%%%%%%%%%%%%%%%%%%%%%%%%%%%%%%%%%%%%%%%%%%%%%%%%%%

%\includeonly{chap1}

%
% Start the Report
%
\begin{document}

\parindent 0pt
\parskip 10pt

%%%%%%% TITLE PAGE %%%%%%%%%%%%%%%%%%%%%%%%%%%%%%%

\thispagestyle{empty}
\vspace*{\fill}
\begin{center}

\pspicture(15.5,0)
\psline[linestyle=solid,linecolor=red,linewidth=2pt](-0.1,0)(15.6,0)
\endpspicture

\vspace*{1mm}
{\Huge\bf{ SEM2DPACK}}\\[5mm]
{\Large A Spectral Element Method tool for 2D wave propagation}\\
{\Large and earthquake source dynamics}\\[2mm] 
{\Large \bf{User's Guide}}

\pspicture(15.5,0)
\psline[linestyle=solid,linecolor=red,linewidth=1.5pt](-0.1,0.2)(15.6,0.2)
\endpspicture

\newfont{\fdunB}{cmdunh10 scaled \magstep2}
\vspace*{3cm}
\fdunB
Version \Version \\[3mm]
\MonthYear \\[3cm]
Jean-Paul Ampuero\\[2mm]
{\small
California Institute of Technology \\
Seismological Laboratory \\
1200 E. California Blvd., MC 252-21 \\
Pasadena, CA 91125-2100, USA \\
E-mail: ampuero@gps.caltech.edu\\
Web: \url{http://web.gps.caltech.edu/~ampuero}\\
\ \ \\
Phone: (626) 395-6958 \\
Fax  : (626) 564-0715 \\
}
\end{center}
\vfill
%\newpage
\thispagestyle{empty}
\hspace*{0pt}
\newpage
\thispagestyle{empty}
\vspace*{\fill}
\centerline{ \copyright\ 2003-2012 Jean-Paul Ampuero}

%%%%%%% TABLE OF CONTENTS %%%%%%%%%%%%%%%%%%%%%%%%%%%%%%%%%%%%%%%%%%%%%%

\newpage
{\parskip 0pt
  \tableofcontents
}
\newpage

%%%%%%% BODY %%%%%%%%%%%%%%%%%%%%%%%%%%%%%%%%%%%%%%%%%%%%%%%%%%%%%
\chapter{Introduction}

\section{Overview}

The SEM2DPACK package is a set of software tools for the simulation
and analysis of 2D wave propagation and dynamic fracture,
with emphasis on computational seismology and earthquake dynamics.
The core of the package is SEM2D, a solver 
for the 2D elastic wave equations and dynamic earthquake rupture 
based on the Spectral Element Method (SEM) with explicit time stepping.
\charef{phys} of this User's Guide summarizes the range of problems
that can be solved with SEM2DPACK.
\secref{semethod} provides some background on the SEM.
The essential properties of the method are its high order accuracy, 
affordable at competitive computational cost, 
its geometrical flexibility to treat realistic, complicated crustal structures,
and its natural treatment of mixed boundary conditions 
such as fault friction.

SEM2DPACK provides tools for each step
of the general flow of a simulation project:
\begin{enumerate}
  \item {\bf Mesh generation}: partition the domain into (deformed) quadrilateral elements.
Whereas no general mesh generation code is included,
SEM2DPACK contains basic meshing utilities for structured and semi-structured grids
and can import unstructured quadrilateral meshes generated externally.
These features are described in \charef{meshing}.
  \item {\bf Mesh quality verification}: check the accuracy, stability and computational cost, applying tools described in \secref{check}.
Return to previous step if needed. 
  \item {\bf Numerical simulation}: run the SEM2D solver.
\charef{sem2d} explains its usage.
  \item {\bf Post-processing}: visualization and analysis of the output.
A number of post-processing and graphic tools are included,
as decribed in \secref{output}.
Outputs are in the form of binary data files, ASCII data files and PostScript figure files.
Scripts are provided for graphic display and analysis on Seismic Unix, Gnuplot and Matlab.
We recommend the usage of the Matlab functions included (see \secref{matlab}).
\end{enumerate}

This is a research code, constantly under development and provided ``as is'', 
and therefore it should {\em not\/} be considered by the user
as a $100\%$ bug-free software package.
We welcome comments, suggestions, feature requests, bug reports (see \secref{support})
and contributions to the code itself (see \secref{contrib}).

\section{History and credits}

The main part of the elastic-isotropic solver was written by
Dimitri Komatitsch as part of his Ph.D. thesis \cite{Kom97} 
under the direction of Prof. Jean-Pierre Vilotte
at the Institut de Physique du Globe de Paris (IPGP).
The elastic-anisotropic solver and several significant
improvements were added by D. Komatitsch
later as part of a research contract with DIA Consultants.
Further functionalities were added by Jean-Paul Ampuero, 
as part of a Ph.D. thesis \cite{Amp02} also directed by Prof. Vilotte at IPGP.
Most of these additional features were motivated by an ECOS-NORD/FONACYT 
research project for the study of the 
seismic response of the valley of Caracas, Venezuela.
That became the version 1.0 of the SEM2DPACK, released in April 2002.

For version 2.0, most of the solver was rewritten 
in preparation for the implementation of higher level functionalities.
%such as multigrid, subcycling, adaptivity and multiscale coupling.
%While the extensive use of object-oriented features of FORTRAN 9x can degrade performance
%this is not critical in 2D simulations. The emphasis has been 
%rather in facilitating code reuse and extension.
Developments for the simulation of earthquake dynamics \cite{Amp02} 
were included in the main branch of SEM2DPACK in October 2003 (version 2.2). 
Spontaneous rupture along multiple non-planar faults can be currently modelled,
with a range of friction laws.

Non-linear, inelastic materials were introduced in March 2008 (version 2.3). 
Damage and visco-plastic rheologies are included especially
for the modeling of earthquake rupture with off-fault dissipation.

The development of SEM2DPACK has been supported
by the SPICE Research and Training Network of the European Commission,
the National Science Foundation (grant EAR-0944288),
the Southern California Earthquake Center
(funded by NSF Cooperative Agreements EAR-0106924, EAR-0529922, 
and USGS Cooperative Agreements 02HQAG0008, 07HQAG0008)
and URS Corporation.

\section{Application examples}
\label{sec:appli}

SEM2DPACK has been utilized in a variety of applications in Earth sciences.
It has contributed to more than 20 publications in seismic wave propagation:

\begin{sitemize}
  \item wave propagation in anisotropic TTI media \shortcite{Dewetal06}
  \item fault reflections from fluid-infiltrated faults \shortcite{Hanetal07}
  \item benchmark for anisotropic wave propagation \shortcite{Pueetal07}
  \item non-linear wave propagation in damaged rocks \shortcite{Lyaetal09} 
  \item modeling marine seismic profiles \shortcite{RobWhiChr09}
  \item surface wave propagation \shortcite{VigCas10,Vigetal11,Vig12,Boa12,Boa13,Boa14} 
  \item wave propagation around a prototype nuclear waste storage tunnel \shortcite{SmiSni10} 
  \item benchmark for wave propagation in heterogeneus media \shortcite{ObrBea11}
  \item structure effects on the wavefield induced by chemical explosions of the Source Physics Experiment \shortcite{Patetal12}
  \item source estimation using seismic interferometry \shortcite{BehSni13}
\end{sitemize}
and in earthquake rupture dynamics:
\begin{sitemize}
  \item rupture on non-planar faults and seismic wave radiation \shortcite{MadAmpAdd06} 
  \item rupture with rate-and-state friction \shortcite{KanLapAmp08} 
  \item benchmark for dynamic earthquake rupture simulation \shortcite{Pueetal09} 
  \item rupture with off-fault plasticity \shortcite{Haretal11,Gabetal13,XuBZAmp12a,XuBZAmp12b,XuBZ13} 
  \item rupture with off-fault continuum damage \shortcite{Xuetal14} 
  \item rupture in heterogeneous media with low velocity fault zones \shortcite{HuaAmp11,Huaetal14b,Huaetal15} 
  \item dynamic rupture model of the 2011 Tohoku earthquake \shortcite{HuaAmp13,Huaetal14a} 
  \item dynamic rupture model of the 2012 off-Sumatra earthquake \shortcite{MenAmp12} 
  \item rupture with velocity-and-state-dependent friction \shortcite{Gabetal12} 
  \item rupture on branched fault systems \shortcite{Xuetal15} 
\end{sitemize}

\section{Download and updates}

SEM2DPACK is hosted by SourceForge at\\
\centerline{\url{http://sourceforge.net/projects/sem2d/}.}\\
All versions of the code can be downloaded from the package repository at\\
\centerline{\url{http://sourceforge.net/projects/sem2d/files/sem2dpack/}.}

Taking full advantage of the convenient features offered by SourceForge 
(subscribe to new release announcements, submit and track bug reports)
requires a SourceForge.net account, which can be created at\\
\centerline{\url{http://sourceforge.net/account/registration/}.}

SEM2DPACK is updated regularly, typically every few months.
To receive email notifications about new releases you must
sign up for the ``Update Notifications" in the project's main page (scroll down a bit):\\
\centerline{\url{http://sourceforge.net/projects/sem2d/}.}

\section{Requirements}

Compiling the solver code requires the \texttt{make} utility
and a Fortran 95 compiler.
The code is being developed with the Intel compiler for Linux.
It works properly with the Intel compiler starting with
version 8.0.046\_pe047.1, so make sure you have a recent version of \texttt{ifort}. 
Other compilers are not being tested on a regular basis,
so please report any related problems.

The solver runs under the Linux operating system. 
In particular, input/output file name conventions are specific to Linux.
Other operating systems have not been tested.

Pre-processing and post-processing tools, including graphic visualization,
are provided for Seismic Unix, Gnuplot, GMT and Matlab.
The included Matlab tools are by far the most complete,
so a Matlab license is highly recommended.
Matlab ``clone'' softwares have not been tested.

\section{Installation}

\begin{enumerate}
\item Uncompress and expand the SEM2DPACK package:
\texttt{tar xvfz sem2dpack.tgz}
\item Go to the source directory: \texttt{cd SEM2DPACK/SRC} 
\item Edit the \texttt{Makefile} according to your FORTRAN 95 compiler,
following the instructions therein.
\item Modify the optimization parameters declared and described in \texttt{SRC/constant.f90}.
\item Compile: \texttt{make}
\item Move to the \texttt{SEM2DPACK/POST} directory, edit the \texttt{Makefile} and compile.
\end{enumerate}
On normal termination you should end up with a set of executable files, among which
\texttt{sem2dsolve}, in \texttt{/home/yourhome/bin/}.

\section{Documentation}
\label{Sec:docs}

Documentation is available through the following resources:
\begin{sitemize}
  \item This User's Manual
  \item The \texttt{EXAMPLES} directory contains several examples, some have a \texttt{README} file
  \item The pre-processing and post-processing tools for Matlab are documented 
through Matlab's help. For instance \texttt{help mesh2d} provides an overview of
the MESH2D utilities, and \texttt{help mesh2d\_wedge} provides detailed documentation
for the wedge meshing function
  \item The \texttt{ToDo} file contains a list of known issues
\end{sitemize}


\section{Support}
\label{Sec:support}

Support for users of SEM2DPACK is available through a {\it tracking system} at \\
\centerline{\url{http://sourceforge.net/tracker/?group_id=182742},}\\
Three separate {\it tracker lists} deal with the following aspects:
\begin{sitemize}
  \item {\it Feature Requests}: requests for implementation of new features
  \item {\it Support Requests}: questions related to the usage of SEM2DPACK
  \item {\it Bugs}: bug reports
\end{sitemize}
Before submitting an issue make sure that: 
\begin{senumerate}
\item you have read the documentation (see \secref{docs}), 
including the Frequently Asked Questions (\charef{faq}).
Suggestions on how to improve the documentation are also welcome.
\item you are running the most recent version of SEM2DPACK. 
Your issue might have been already fixed in a more recent version.
\item you understand the changes listed in SEM2DPACK's \texttt{ChangeLog} file,
especially changes in the format of the input files
\item your problem has not been treated in previous submissions.
You can browse the tracker message titles or search for keywords.
%By default the browser shows only pending issues, 
%to see the complete list (including resolved issues) set ``Status" to ``Any".
\end{senumerate}
A new submission must include the input files needed to reproduce your problem 
(\texttt{Par.inp}, \texttt{*.ftq}, \texttt{*.mesh2d}, etc). 
You will receive email notifications of any update of your submitted item, until it is closed.
If the item is declared ``Pending'' you are expected to reply to the last message
of the developer within two weeks, otherwise the item will be closed.
For more instructions see\\
\centerline{{\url{http://sourceforge.net/support/getsupport.php?group_id=182742}.}}

\section{Contributions}
\label{Sec:contrib}

Contributions to SEM2DPACK by experienced programmers are always welcome and encouraged.
Although the code is stable for typical applications in computational seismology and
earthquake dynamics, there is still a number of missing features.
Their implementation could make SEM2DPACK interesting for a broader audience in mechanical engineering, 
geotechnical engineering, applied geophysics and beyond. 

The solver code is written in FORTRAN 95.
Resources available for programmers include:
\begin{sitemize}
  \item A \texttt{ToDo} file included with SEM2DPACK
contains a wish list that ranges from basic functionalities to complex code re-engineering.
  \item \charef{add} gives some guidelines for programmers.
  \item A ``Developers Forum" to discuss the implementation of new features is available at \\
\centerline{\url{http://sourceforge.net/forum/forum.php?forum_id=635737},}
\end{sitemize}

We acknowledge the following contributions:
\begin{sitemize}
  \item Shiqing Xu (USC): contributed to the damage material module
  \item Yihe Huang (Caltech): Gaussian source time function, visco-elastic material (attenuation)
  \item Trevor Currie (Caltech): rate-and-state friction and earthquake cycle solver 
% Alice : matlab scripts?
\end{sitemize}

\section{Citing SEM2DPACK in your publications}

If you use SEM2DPACK in academic publications, please properly acknowledge it in your article.
Cite the information below or this user’s manual in the references list, main text, 
acknowledgements section or methods section, according to the journal policies:

"SEM2DPACK - A Spectral Element Method tool for 2D wave propagation and earthquake source dynamics, 
version xxxx, by J. P. Ampuero, \url{http://sourceforge.net/projects/sem2d/}". 

In addition, cite \cite{Amp02} or
one of the articles co-authored by J. P. Ampuero listed in section \ref{sec:appli}.

\section{License}
\input{../../LicenseNotice}

\chapter{Physical background}
\label{Cha:phys}

This chapter summarizes the physical assumptions and notations in SEM2DPACK.
Footnotes provide reference to the input arguments described in \charef{sem2d}.

\section{General assumptions and conventions}

The coordinate sytem is Cartesian (rectangular). 
SEM2DPACK works in the two-dimensional $(x,z)$ plane,
where $x$ is the horizontal coordinate, with positive direction pointing to the right, 
and $z$ is the vertical coordinate, with positive direction pointing upwards.
The coordinates $(x,y,z)$ will be also denoted as $(x_1,x_2,x_3)$.
This notations carry also for subscripts. For instance,
the $i$-th component of displacement is denoted as $u_i$, with $i=1,2,3$ or with $i=x,y,z$.

The reference frame is Eulerian.
Infinitesimal strain is assumed.
The (symmetric) infinitesimal strain tensor $\varepsilon$ is defined as 
\eq
 \varepsilon_{ij} = \frac{1}{2} \left(\frac{\partial u_i}{\partial x_j} +\frac{\partial u_j}{\partial x_i}\right)
\en
%Dynamic fault boundary conditions assume also small displacements....?

Material density is deonted $\rho(x,z)$.
The displacements and stresses relative to an initial equilibrium configuration
are denoted $u_i(x,z,t)$ and $\sigma_{ij}(x,z,t)$, respectively.
External forces (sources) are denoted $f_i(x,z,t)$. 
SEM2DPACK solves the following equations of motion to obtain
the relative displacements $u_i(x,z,t)$:
\eq
  \rho\ \frac{\partial^2 u_i}{\partial t^2} = \frac{\partial\sigma_{ij}}{\partial x_j} + f_i
\en
where summation over repeated indices is assumed.
The initial conditions are $u_i=0$ and $\partial u_i/\partial t = 0$.
Stresses are related to strain, and possibly to other internal variables,
by constitutive equations described in \secref{rheol}.
The governing equations are supplemented by boundary conditions, described in \secref{bcs}. 
SEM2DPACK actually solves the governing equations in variational (weak) form,
as described in any textbook on the finite element method.

Two types of 2D problems are solved\footnote{In the \texttt{\&GENERAL} input block,
plane strain is selected by $\texttt{ndof=2}$ and antiplane shear by $\texttt{ndof=1}$.}: 
\begin{sitemize}
\item \emph{Plane strain}: 
Also known in seismology as P-SV, and in fracture mechanics as inplane mode or mode II.
It is assumed that $u_3=0$ and $\partial / \partial x_3 = 0$.
Hence, $\varepsilon_{13}=\varepsilon_{23}=\varepsilon_{33}=0$
and there are two degrees of freedom per node, $u_x$ and $u_z$.
\item \emph{Antiplane shear}: 
Also known in seismology as SH, and in fracture mechanics as antiplane mode or mode III.
It is assumed that $u_1=u_2=0$ and $\partial / \partial x_3 = 0$. 
Hence, only $\varepsilon_{13}$ and $\varepsilon_{23}$ are non-zero
and there is one degree of freedom per node, $u_y$.
\end{sitemize}

\section{Material rheologies}
\label{Sec:rheol}

We describe here the constitutive equations implemented in SEM2DPACK,
relating stress ($\sigma_{ij}$), strain ($\varepsilon_{ij}$) and internal variables.

\subsection{Linear elasticity}

\subsubsection{Linear isotropic elasticity}

Stress and strain are linearly related by Hooke's law, 
$\sigma_{ij} = c_{ijkl} \varepsilon_{ij}$,
where $c_{ijkl}$ is the tensor of elastic moduli.
In particular, for isotropic elasticity:
\eq
  \sigma_{ij} = \lambda \varepsilon_{kk} \delta_{ij} + 2\mu \varepsilon_{ij}
\en
where $\lambda$ and $\mu$ are Lam\'e's first and second parameters, respectively.
In 2D plane strain the only relevant stress components are $\sigma_{11}$, $\sigma_{22}$ and $\sigma_{12}$.
The intermediate stress $\sigma_{33}$, although not null, does not enter in the equations of motion.
The S and P wave speeds are $c_S = \sqrt{\mu/\rho}$ and $c_P = \sqrt{(\lambda+2\mu)/\rho}$, respectively.
In 2D antiplane shear only the stress components $\sigma_{13}$ and $\sigma_{23}$ are relevant,
and only S waves are generated.

\subsubsection{Linear anisotropic elasticity}

Transverse anisotropy with vertical symmetry axis (VTI)
is implemented for 2D P-SV \shortcite{KoBaTr00b}. % and SH modes.
The stress-strain constitutive relation for P-SV in Voigt notation is:
\eq
  \left( \begin{array}{c}
    \sigma_{xx} \\ \sigma_{zz} \\ \sigma_{xz}
  \end{array} \right)
  = 
  \left( \begin{array}{ccc}
    c_{11} & c_{13} & 0 \\
    c_{13} & c_{33} & 0 \\
    0      & 0      & c_{55}
  \end{array} \right) \ 
  \left( \begin{array}{c}
    \varepsilon_{xx} \\ \varepsilon_{zz} \\ 2\varepsilon_{xz}
  \end{array} \right)
\en
where the $c_{ij}$ are elastic moduli. 
For SH:
\eq
  \left( \begin{array}{c}
    \sigma_{yz} \\ \sigma_{xy} 
  \end{array} \right)
  = 
  \left( \begin{array}{cc}
    c_{55} & 0 \\
    0 & c_{66} 
  \end{array} \right) \ 
  \left( \begin{array}{c}
    2 \varepsilon_{yz} \\ 2\varepsilon_{xy}
  \end{array} \right)
\en


\subsection{Linear visco-elasticity}

\subsubsection{Generalized Maxwell material}
\emph{Not implemented yet.}

\subsubsection{Kelvin-Voigt material}

Kelvin-Voigt damping can be combined with any of the other constitutive equations
by replacing the elastic strain $\varepsilon$ by $\varepsilon^* = \varepsilon + \eta\,\partial\varepsilon/\partial t$,
where $\eta$ is a viscosity timescale.

The resulting quality factor $Q$ is frequency-dependent, $Q^{-1}(f) = 2\pi\eta f$.
This rheology is not approriate to model crustal attenuation with constant $Q$, 
unless the source has a narrow frequency band
and $\eta$ is selected to achieve a given $Q$ value at the dominant frequency of the source.

The main application of Kelvin-Voigt viscosity is the artificial damping of high-frequency
numerical artifacts generated by dynamic faults.
Dynamic source simulations using methods that discretize the bulk, such as
finite difference, finite element and spectral
element methods, are prone to high frequency numerical noise 
when the size of the process zone is not well resolved.
Efficient damping is typically achieved
by a thin layer of Kelvin-Voigt elements surrounding the fault,
with thickness of 1 or 2 elements on each side of the fault
and $\eta/\Delta t_{fault} = 0.1$ to $0.3$, where $\Delta t_{fault}$ 
is the critical time step size based on the size of the spectral elements along the fault
(not necessarily equal to the critical time step over the whole mesh).
The value of $\Delta t_{fault}$ can be obtained with 
the Matlab function \texttt{PRE/critical\_timestep.m}.

%Methods to control this problem were presented in the author's
%Ph.D. dissertation \cite{Amp02}\footnote{Available in French at 
%\url{http://web.gps.caltech.edu/~ampuero/publications.html}}
%and in Gaetano Festa's Ph.D. dissertation\footnote{\url{http://people.na.infn.it/~festa/}},
%%In particular the method of Appendix 4.A, a high/low frequency 
%%decomposition with hybrid and consistent asymptotic/SEM treatment of friction,
%%has been tested by Gaetano Festa and J.-P. Vilotte
%and will be implemented in a forthcoming version of SEM2DPACK.

\subsection{Coulomb plasticity and visco-plasticity}

\subsubsection{Perfect plasticity}

Perfect plasticity with a Coulomb yield function is implemented
for 2D plane strain, as in \citeN{And05}.

The total strain is the sum of an elastic and a plastic contribution,
$\varepsilon = \varepsilon^e + \varepsilon^p$.
The plastic strain is assumed to be purely deviatoric ($\varepsilon^p_{kk} = 0$).
Plastic yield occurs when
the maximum shear stress over all orientations,
\eq
  \tau_{max} = \sqrt{ \sigma_{xz}^2 + (\sigma_{xx}-\sigma_{zz})^2/4},
\en
reaches the yield strength,
\eq
  Y = c\,\cos(\phi) - (\sigma_{xx}+\sigma_{zz}) \, \sin(\phi)/2,
\en
where $c$ is the cohesion and $\phi$ is the internal friction angle.

\subsubsection{Visco-plasticity}

In classical Duvaut-Lions visco-plasticity the (visco-)plastic strain rate is proportional
to the excess of stress over the yield strength:
\eq
  \dot{\varepsilon}^p_{kl} = \frac{1}{2\mu\ T_v} \langle\tau_{max} - Y\rangle\ \frac{\tau_{ij}}{\tau_{max}}
\en
where $T_v$ is the visco-plastic relaxation time,
$\langle x\rangle \doteq (x+|x|)/2$ is the ramp function
and $\tau_{ij} = \sigma_{ij} - \frac{1}{3}\sigma_{kk}\ \delta_{ij}$ 
is the deviatoric part of the stress tensor.

Visco-plasticity is often employed as a regularization of plasticity
to avoid or delay the occurrence of strain localization features,
such as shear bands, that are mesh-dependent.
For that particular application, $T_v$ is typically set to 
the average P wave traveltime across a few grid points, 
i.e. a few times the average spacing between GLL nodes divided by the P wave speed.

%----------------------------------------------------------
\subsection{Attenuation}
Following \shortciteN{Mocetal04}, we incorporate attenuation 
by adding viscoelastic terms in the stress-strain relations.
For the sake of simplicity, we present here the formulation in terms of scalar stress and strain:
\eqa
\sigma(t) &=& M_u \varepsilon(t)-\sum_{l=1}^n M_u Y_l \zeta_l(t) \\
\dot{\zeta}_l(t)+\omega_l \zeta_l(t) &=& \omega_l \varepsilon_l(t)
\label{Eq:Qf}
\ena
where $M_u$ is the unrelaxed modulus, $n$ is the number of anelastic mechanisms
and $Y_l$, $\zeta_l(t)$ and $\omega_l$ are the anelastic coefficient, 
anelastic state variable and relaxation frequency of the $l^{th}$ viscoelastic mechanism, respectively. 

The material parameters $Y_l$ and $\omega_l$ are determined to achieve
a prescribed constant $Q$ value within a prescribed frequency band $[\omega_1,\omega_2]$.
The quality factor of this viscoleastic material, as a function of angular frequency $\omega$, is
implicitely given by the solution of the following equation:
\eq
Q^{-1}(\omega)- \sum_{l=1}^n \frac{\omega_l \omega +\omega_l^2 Q^(-1)(\omega)}{\omega_l^2+\omega^2 } Y_l = 0
\en
The relaxation frequencies $\omega_l$ are distributed uniformly in logarithmic scale
within $[\omega_1,\omega_2]$.
The anelastic coefficients $Y_l$ are determined by satisfying Equation~\ref{Eq:Qf} with prescribed $Q$ 
in a least squares sense at $2n-1$ frequencies 
distributed uniformly in logarithmic scale within $[\omega_1,\omega_2]$.
The unrelaxed moduli $M_u$ are derived from the anelastic coefficients, 
following equations 166 and 167 of \shortciteN{Mocetal04}),
by constrainging the phase velocity at the reference frequency $\omega_r=\sqrt{\omega_1 \omega_2}$
to equal the prescribed elastic wave velocity.

For $n=3$, an almost constant $Q$ is achieved, with less than $5\%$ error, 
over a frequency band with maximum to minimum frequency ratio $\sim 100$. 
A benchmark of our attenuation implementation is presented in section 5.1 of \shortciteN{Huaetal14b}.

%----------------------------------------------------------
\subsection{Continuum damage}

The continuum damage formulation by \shortciteN{Lyaetal97a}, 
including damage-related plasticity as introduced by \shortciteN{Hametal04},
is implemented with modifications for 2D plane strain.
%and XXX $\beta$ as introduced by \shortciteN{Hametal04}.

The first and second invariants of the 2D elastic strain tensor are defined as
$I_1 = \varepsilon^e_{kk}$ and $I_2 = \varepsilon^e_{ij}\varepsilon^e_{ij}$, respectively. 
A strain invariant ratio is defined as $\xi = I_1/\sqrt{I_2}$.
The following non-linear stress-strain relation is assumed
\shortcite[eq. 12]{Lyaetal97a}:
\eq
  \sigma_{ij} = ( \lambda - \gamma/\xi )\ I_1 \delta_{ij} 
              + ( 2\mu - \gamma \xi )\ \varepsilon^e_{ij}
\en
where $\gamma$ is an additional elastic modulus.
The elastic moduli depend
on a scalar damage variable, $0\le\alpha\le1$, through \shortcite[eq. 19]{Lyaetal97a}: 
%(\shortciteN[eq. 19]{Lyaetal97a} and \shortciteN[eq. 3]{Hametal04}): % for beta>0
\eqa
  \lambda &=& \lambda_0 \\
  \mu &=& \mu_0 + \gamma_r \xi_0\ \alpha \\
  %\gamma &=& \gamma_r\ \frac{\alpha^{1+\beta}}{1+\beta}
  \gamma &=& \gamma_r\ \alpha
\ena
where $\lambda_0$ and $\mu_0$ are Lam\'e's parameters for the intact material ($\alpha=0$).
The parameter $\xi_0$ is the threshold value of the strain invariant ratio $\xi$
at the onset of damage.
It is related to the internal friction angle $\phi$ in
a cohensionless Mohr-Coulomb yield criterion by 
the 2D plane strain version of \shortciteN[eq. 37]{Lyaetal97a}:
\eq
  \xi_0 = \frac{-\sqrt{2}}{\sqrt{1+ (\lambda_0/\mu_0 +1)^2\ \sin^2\phi}} % in 2D
  %\xi_0 = \frac{-\sqrt{3}}{\sqrt{ 2q^2 (\lambda_0/\mu_0 +2/3)^2 +1}}    % in 3D
\en
%where
%\eq
%  q = \frac{\sin(\phi)}{1-\sin(\phi)/3}
%\en
%The parameter $0<\beta<1$ was introduced by \shortcite{Hametal04}.
The scaling factor $\gamma_r$ is chosen such that convexity is lost
at $\alpha=1$ when $\xi=\xi_0$.
It is derived from the 2D plane strain version of \shortciteN[eq. 15]{Lyaetal97a}:
\eq
 \gamma_r = p+\sqrt{p^2 +2\mu_0 q}
\en
where
\eqa
 q & =& (2\mu_0+2\lambda_0)/(2-\xi_0^2) \\  % in 2D
 %q & =& (2\mu_0+3\lambda_0)/(3-\xi_0^2) \\ % in 3D
  p &=& \xi_0 (q + \lambda_0)/2 
\ena
The evolution equation for the damage variable is
\shortcite[eq. 20]{Lyaetal97a}
%(\shortciteN[eq. 20]{Lyaetal97a} and \shortciteN[eq. 7]{Hametal04})
\eq
  \dot{\alpha} = 
   \left\{
   \begin{array}{ll}
     %C_d I_2 (\alpha^\beta \xi - \xi_0)& \mbox{if $\xi>\xi_0/\alpha^\beta$} \\
     C_d I_2 (\xi - \xi_0)& \mbox{if $\xi>\xi_0$} \\
     0  & \mbox{otherwise}
   \end{array}
  \right.
\en
No healing is assumed below $\xi_0$.
The evolution of the plastic strain $\varepsilon^p_{ij}$ is 
driven by the damage variable $\alpha$ 
\shortcite[eq. 9]{Hametal04}:
\eq
  \dot{\varepsilon}^p_{ij} = 
   \left\{
   \begin{array}{ll}
     \tau_{ij} C_v \dot{\alpha} & \mbox{if $\dot{\alpha}\ge 0$} \\
     0  & \mbox{otherwise}
   \end{array}
  \right.
\en
where $\tau_{ij} = \sigma_{ij} - \frac{1}{3}\sigma_{kk}\ \delta_{ij}$ 
is the deviatoric part of the stress tensor.
The parameter $C_v$ is of order $1/\mu_0$
and\footnote{In \texttt{\&MAT\_DMG}, the input argument \texttt{R} is defined as $R=\mu_0 C_v$.}
is related to the seismic coupling coefficient $0<\chi<1$ by 
\cite{BZLya06}
\eq
  C_v = \frac{1-\chi}{\chi}\ \frac{1}{\mu_0}
\en
%and the factor $R = \mu_0 C_v$ is related to the seismic coupling coefficient $0<\chi<1$ by $\chi=1/(1+R)$
%$\chi=1/(1+\mu_0 C_v)$

%----------------------------------------------------------
%----------------------------------------------------------
\section{Boundary conditions}
\label{Sec:bcs}

\subsection{Absorbing boundaries}

Two approximate absorbing boundary conditions (ABC)
are implemented to model the outwards propagation of waves 
at the boundaries of the computational domain.
Both conditions are of paraxial type. Their performance is appropriate at normal incidence
but degrades at grazing incidence. 

\subsubsection{Clayton-Engquist ABC}

In the local coordinate frame $(t,n)$ related to the tangential ($t$)
and outgoing normal ($n$) directions to the boundary, the
first-order accurate ABC proposed by \citeN{ClaEng77} reads:
\eqa
  \dot{u}_t &=& - c_S \frac{\partial u_t}{\partial x_n} \\
  \dot{u}_n &=& - c_P \frac{\partial u_n}{\partial x_n}
\ena
The implementation is based on an equivalent formulation 
as a mixed boundary condition
that relates tractions $T$ to displacements $u$:
\eqa
  T_t &=& - \rho c_S\ \dot{u}_t \\
  T_n &=& - \rho c_P\ \dot{u}_n
\ena
The formulation above is for P-SV mode. 
In SH mode the ABC is $T_y= - \rho c_S\ \dot{u}_y$.

\subsubsection{Stacey ABC}

The second-order accurate ABC introduced by \citeN{Sta88} 
under the name ``P3'' is:
\eqa
  \dot{u}_t &=& - c_S \frac{\partial u_t}{\partial x_n}
                - (c_P-c_S) \frac{\partial u_n}{\partial x_t}\\
  \dot{u}_n &=& - c_P \frac{\partial u_n}{\partial x_n}
                - (c_P-c_S) \frac{\partial u_t}{\partial x_t}
\ena
Its formulation as a mixed boundary condition is:
\eqa
  T_t &=& - \rho c_S\ \dot{u}_t + \rho c_S (2 c_S-c_P)\ \frac{\partial u_n}{\partial x_t}\\
  T_n &=& - \rho c_P\ \dot{u}_n - \rho c_S (2 c_S-c_P)\ \frac{\partial u_t}{\partial x_t}
\ena
This ABC is only implemented in P-SV mode.

\section{Fault interface conditions}

\subsection{Linear slip law}
Represents a compliant fault zone with elastic contact.
See \shortciteN{Hanetal07}.
[...]

\subsection{Normal stress response}

\subsubsection{Unilateral contact}

No interpenetration during contact, free stress during opening.
[...]

\subsubsection{Modified Prakash-Clifton regularization}

Regularization of the normal stress response, as required for
bimaterial rupture problems, is implemented following \citeN{RubAmp07}.
The frictional strength is proportional to a modified normal stress $\sigma^*$,
related to the real fault normal stress, $\sigma$,
by either of the following evolution laws:
\begin{sitemize}
 \item Version with a regularization \emph{time} scale:
\eq
  \dot{\sigma^*} = \frac{1}{T_\sigma}\ (\sigma - \sigma^*)
\en
where $T_\sigma$ is a constitutive parameter\footnote{In \texttt{\&BC\_DYNFLT\_NOR},
this law is set by \texttt, and the relevant parameter is \texttt{T}.}.
 \item Version with a regularization \emph{slip} scale:
\eq
  \dot{\sigma^*} = \frac{|V| + V^*}{L_\sigma}\ (\sigma - \sigma^*)
\en
where $V$ is slip rate, 
and $V^*$ and $L_\sigma$ are constitutive parameters\footnote{In \texttt{\&BC\_DYNFLT\_NOR},
this law is set by \texttt, and the two relevant parameters are \texttt{V} and \texttt{L}.}.
\end{sitemize}

\subsection{Friction}

\subsubsection{Slip-weakening friction}
Slip occurs when the fault shear stress reaches the shear strength $\tau = \mu\sigma$ 
(or $\tau=\mu\sigma^*$ if the Prakash-Clifton law is assumed). [...]
The friction coefficient $\mu$ is a function of the cumulated slip $D$,
given by one of the following laws:
\begin{sitemize}
  \item Linear slip-weakening law:
\eq
  \mu = \max\left[ \mu_d , \mu_s - \frac{\mu_s-\mu_d}{D_c}\ D \right]
\en
  \item Exponential slip-weakening law:
\eq
  \mu = \mu_s - (\mu_s-\mu_d)  \exp(-D/D_c)
\en
\end{sitemize}

\subsubsection{``Fast'' rate-and-state-dependent friction}

Friction with fast (power law) velocity weakening at fast slip speed
is a first order proxy for physical weakening processes that operate
on natural fault zones at coseismic slip velocities.
A rate-and-state dependent friction law with fast velocity-weakening is implemented in SEM2DPACK,
similar to that adopted e.g. by \citeN{AmpBZ08}.
The friction coefficient depends on slip velocity ($V$) and a state variable ($\theta$):
\eq
  \mu_f = \mu_s +a\ \frac{V}{V+V_c} - b\ \frac{\theta}{\theta+D_c}.
  \label{Eq:muf}
\en
The state variable has units of slip and is governed by the evolution equation
\eq
  \dot{\theta} = V-\theta V_c/D_c.
\label{Eq:th}
\en
The friction law is defined by the following constitutive parameters:
$\mu_s$ is the static friction coefficient,
$a$ and $b$ are positive coefficients of a direct effect and an evolution effect, respectively,
$V_c$ is a characteristic velocity scale\footnote{\texttt{vstar} in \texttt{\&BC\_DYNFLT\_RSF}},
and $D_c$ is a characteristic slip scale.

The steady-state ($\dot{\theta}=0$) friction coefficient 
\eq
  \mu_f = \mu_s +(a-b)\ \frac{V}{V+V_c}
\en
weakens asymptotically as $1/V$ when $V\gg V_c$,
if $a<b$, approaching its dynamic value ($\mu_d = \mu_s + a-b$)
over a relaxation timescale $D_c/V_c$.
The value of the relaxation time $D_c/V_c$ tunes the weakening mechanism between two limit cases:
slip-weakening and velocity-weakening.
If $D_c/V_c$ is much longer than the typical time scale of fluctuation of the
state variable ($\approx \theta/\dot{\theta}$),
\Eqref{th} becomes $\dot{\theta} \approx V $,
implying that $\theta$ is proportional to slip
and that the evolution term of the friction coefficient
is effectively slip-weakening, with characteristic slip-weakening distance $D_c$.
Conversely, if $D_c/V_c$ is short the relaxation to steady state is fast,
$\theta/D_c \approx V/V_c$
and the friction is effectively velocity-weakening,
with characteristic velocity scale $V_c$.

\subsubsection{Logarithmic rate-and-state friction}
\emph{Not implemented yet.}

Dieterich and Ruina classical rate-and-state laws, with aging or slip state evolution law.

\chapter{Mesh generation}
\label{Cha:meshing}

\section{General guidelines}

The Spectral Element Method (SEM) requires an initial decomposition 
of the space domain into quadrilateral elements (a quad mesh).
Obtaining the best performance (accuracy/cost) out of the SEM imposes constraints
on the mesh design:
\begin{sitemize}
\item The interfaces between different materials, 
at which sharp contrasts of material properties occur,
should {\it preferably} coincide with faces of the elements. 
This is sometimes called an {\it adapted mesh} 
and is the only way to preserve spectral accuracy at material interfaces.
\item Fault planes, across which displacement discontinuities occur,
{\it must} coincide with element faces. 
Faults are implemented with a {\it split node} formulation.
\item Elements can be deformed, but extremely small and extremely large 
angles between faces of a same element must be avoided. 
This would penalize both accuracy and stability.
\item The linear size of the elements must be small enough, so that each element
contains enough computational nodes per minimum wavelength,
and each fault boundary element contains enough nodes per rupture process zone.
\item Unnecessarily small elements should be avoided, 
they penalize the stability of the method. 
\end{sitemize}
Generating high quality quad meshes for complicated geological models
is not yet a fully automated process, and can be very time-consuming.
Iterations between mesh generation and mesh quality check are sometimes required.
The last two constraints above are addressed more quantitatively in \secref{check}.
Mesh quality assessment tools are also presented in \secref{check}.

The remainder of this chapter describes three possible ways to generate quad meshes,
by order of complexity:
\begin{senumerate}
\item If the geometrical structure of the model is simple
or if the user prefers to sacrifice accuracy by using a non-adapted structured mesh,
i.e. a logically cartesian mesh 
where the element faces do not necessarily follow the material interfaces,
the basic built-in meshing capabilities of the solver SEM2D, described
in \secref{mesh_inc}, are sufficient.
\item Moderately complicated meshes can be generated with the included Matlab tools,
described in \secref{mesh2d}.
\item Adapted meshes for more complicated geological models 
must be generated with some external software.
As an illustration, the usage of the freely available 2D mesh generation software EMC2 
is described in \secref{emc2}.
\end{senumerate}

SEM2DPACK provides only basic meshing capabilities
and does not include an unstructured mesh generator 
for complicated, realistic geological models. 
This chapter describes how to achieve that task with an external software, EMC2.

Generating a high quality unstructured quad mesh can be a time-consuming task.
Let's note that, for wave propagation problems without dynamic faults, 
if the acceptable accuracy is low
(or large computational resources are available to work with a very fine mesh)
a structured mesh in which the element faces do not necessarily follow the material interfaces
can be generated with the basic built-in meshing capabilities of SEM2D.

\section{Meshing features included in the solver}
\label{Sec:mesh_inc}

The solver itself has very limited meshing capabilities.
It can only generate a structured mesh for a single quadrilateral domain,
possibly with curved sub-horizontal boundaries and curved sub-horizontal layer interfaces.
The domain can be cut in the horizontal direction by a single fault, possibly curved or kinked.

For further details see the Reference Guide for the input blocks
\texttt{MESH\_CART} and  \texttt{MESH\_LAYERED} in \secref{inblo}.

\section{Meshing with the MESH2D Matlab utilities}
\label{Sec:mesh2d}

A number of Matlab functions for 2D meshing are provided in \texttt{POST/mesh2d}.
These can generate structured meshes for quadrilateral domains
with curved boundaries, and merge several such meshes to generate
a more complicated, globally unstructured mesh.
Functions for manipulating, visualizing and exporting these meshes are included.
Here is an overview of available tools:

\begin{verbatim}
  SEM2DPACK/PRE/mesh2d provides Matlab utilities for the generation, manipulation 
  and visualization of structured 2D quadrilateral meshes, and unstructured 
  compositions of structured meshes.
 
  The mesh database is stored in a structure described in MESH2D_TFI's help.
 
  Mesh generation:
 
   MESH2D_TFI     	generates a structured mesh by transfinite interpolation
   MESH2D_QUAD    	generates a structured mesh for a quadrilateral domain
   MESH2D_CIRC_HOLE	generates a mesh for a square domain with a circular hole 
   MESH2D_WEDGE		generates a mesh for a triangular wedge domain
   MESH2D_EX0		mesh for a vertical fault
   MESH2D_EX1		mesh for a shallow layer over half-space with dipping fault
 
  Mesh manipulation:
 
   MESH2D_ROTATE   	rotates the node coordinates 
   MESH2D_TRANSLATE	translates the node coordinates
   MESH2D_MERGE    	merges several meshes into a single mesh
   MESH2D_WRITE  	writes a 2D mesh database file (*.mesh2d)
   MESH2D_READ 		reads a 2D mesh database from a *.mesh2d file 
 
  Reading mesh files from other mesh generation software:
 
   READ_DCM 		reads a 2D mesh in the DCM format of EZ4U (http://www-lacan.upc.es/ez4u.htm)
   READ_INP		reads a 2D mesh in the INP format of ABAQUS exported by CUBIT
 
  Mesh visualization:
 
   MESH2D_PLOT        	plots a 2D mesh
 
  Miscellaneous tools:
 
   SAMPLE_SEGMENTS	generates points that regularly sample multiple segments of a line
\end{verbatim}

The functions \texttt{MESH2D\_TFI} and \texttt{MESH2D\_MERGE} are the core tools.
The script \texttt{MESH2D\_EXAMPLE1} is a good starting point.
The syntax of the mesh database file, \texttt{*.mesh2d}, is described
in \secref{inblo}.

\section{Generating a mesh with EMC2}
\label{Sec:emc2}

\subsection{The mesh generator EMC2}

EMC2 is one of the few public domain 2D mesh generation softwares 
that includes quadrilateral elements and a Graphical User Interface. 
Its C code sources and executables can be freely downloaded from\\
%\url{http://www-c.inria.fr/gamma/cdrom/www/emc2/eng.htm}.
\centerline{\url{http://www.ann.jussieu.fr/~hecht/ftp/emc2/}.}\\
We show here an example featuring the most useful functionalities of EMC2.
For further details you should refer to the complete documentation of
the EMC2 package,

Before starting you must prepare files containing in 2-column data 
the coordinates (X,Z)
of all the points needed to define the geometry of the model 
(topography, sediment bottom). 
%An example is given in \texttt{SEM2DPACK/EMC2SEM/palosgrandes.dat}.

Once installed, you can run EMC2 by typing \texttt{emc2}.

\subsection{Notations}

The following notations are assumed in the next section:
\begin{itemize}
\item  \textsf{(XXX)} = click XXX on top menu bar 
\item  \textsf{(xxx)} = click xxx on bottom menu bar
\item  \textsf{$<$XXX$<$} = click XXX on left menu bar
\item  \textsf{$>$XXX$>$} = click XXX on right menu bar
\item  \textsf{\$xxx\$} = enter xxx from keyboard or from the calculator in the right panel
\item  \textsf{"xxx"} = type xxx in bottom prompt
\item  \textsf{\{xxx\}} = perform action xxx
\item  \textsf{*xxx}  = do xxx as many times as needed
\item  \textsf{n*xxx} = do xxx n times
\end{itemize}

\subsection{Basic step-by-step}

A typical EMC2 session has three steps:

\begin{enumerate}[STEP I:]
\item CONSTRUCT\label{cons}, defines the geometry of the model

\begin{enumerate}[1.]
\item Switch to the construction tool:\\
      \textsf{$<$CONSTRUCTION$<$}
\item Load the points:\\
      \textsf{(POINT) (xy file) "palosgrandes.dat"}\\
   You must give the full path to your points-file,
   the root directory being the one where you launched \texttt{emc2}.
   \label{loadpoints} 

\item Reset the figure window to fit all points: \\
	\textsf{$>$SHOW ALL$>$}\\

The original data has some geometrical features 
that are too complex to be meshed by quadrilaterals, for instance
the corners at the N and S ends of the basin, you may
want to smooth out these features. 
You also need to define the extreme 
boundaries of the region to be modelled (N,S and bottom
absorbing boundaries) and some additional points on
the free surface outside the basin.
You must modify the data set (add and delete points):

\item Add new points:

  \begin{enumerate}[a.]
  \item with the mouse:\\
	\textsf{(POINT) (mouse) *\{click in figure window\}}

  \item by coordinates:\\
	\textsf{(POINT) (xy pt) *\{ \$x=y=\$ \}}\\
     This is the safest way to get really vertical and
     horizontal boundaries needed for the absorbing conditions
     in SPECFEM90. You probably need to get
     the coordinates of an existing reference point:\\
        \textsf{(POINT) $<$QUERY$<$ (point) *\{click on point\}}

  \item you can also reload another point-file (\ref{loadpoints})

  \end{enumerate}

\item Delete points, \\
	\textsf{(POINT) $<$DESTRUCT$<$ (point) *\{click on point\}}\\

Now you must define the geometry of the domains. These
macro-blocks are intended to be internally meshed by 
deformed quadrilaterals. Their geometry follows the geometry
of the geological model (one domain per material).
Each domain must be bounded by segments or splines:

\item Segments:\\
	\textsf{(SEGMENT) (point) 2*\{click extreme point\}}

\item Splines:\\
        \textsf{(SPLINE) (point) *\{click point\}}\\
   You will see the spline evolve as you click points.

\end{enumerate}


\item PREPARE, defines the properties of the discrete spectral element mesh

  \begin{enumerate}[1.]

  \item Switch to the preparing mesh tool:\\
	\textsf{$<$PREP MESH$<$}

  \item Define domains with rock n:\\
	\textsf{(DOMAIN REF) \$n=\$ (any) *\{click inside domain\}}\\
   You will see the domains edges get colored and the 
   domains get numbered with n.

  \item At any moment you can decide to show or not the
domain decomposition:\\ 
  To hide the domain decomposition:\\
        \textsf{$>$REFRESH$>$}\\
  Show the domain decomposition:\\
	\textsf{(SHOW) (ALL)}

  \item Remove a domain definition:\\
	\textsf{(REMOVE) (DOMAINE) (any) \{click inside domain\}}\\
   WARNING: corrections to the domain decomposition are
   sometimes displayed only after refreshing the figure window.

  \item Now you must define the subdivision of each 
   domain in quadrilateral finite elements. 
   Define the number $n$ of elements on each edge:\\
	\textsf{(NB INTERVAL) \$n=\$ (any) \{click edge\} }\\
   You will see the intermediate points appear.
   The number of intervals $n$ is mainly dictated by the resolution criterion: 
   elements should be smaller than the smallest wavelength you want to propagate.
   Moreover, {\bf a domain can be quadrangulated only if
   the total number of intervals along its perimeter is even} 
   (the sum of all $n$ along its boundaries).
   However, a quality mesh is not always guaranteed
   and you need to proceed by trial and error (emc2 allows you
   to jump back and forth between the different steps of the
   meshing procedure).

\item Finally you must define the external boundaries of 
the modelled region which will have a special treatment.
You must associate a tag (a number) to each absorbing boundary.
No convention is assumed but you should remember those tags later
when setting the boundary conditions in SEM2D. It is also useful
to assign a tag to the free surface boundary, that will be eventually
used by SEM2D to locate the receivers or sources.

   Define a boundary with index n:\\
	\textsf{(LINE REF) \$n=\$ (any) *\{click edge\}}\\ 
   Of course each boundary can be composed of many 
   domain edges.
   Refresh the display to better see the boundaries.
   \emph{The same procedure applies to define split-node interfaces such as faults and cracks: 
   you must assign a different tag to each side of the fault.}

  \item Save your work in EMC2 format: \\
	\textsf{$<$SAVE$<$ "name"}\\
   The resulting file is \texttt{name.emc2\_bd}
   
  \end{enumerate}

\item EDIT, generates the mesh

  \begin{enumerate}[1.]

  \item Switch to the edit mesh tool: \\
	\textsf{$<$EDIT\_MESH$<$} \\
   Press ENTER 4 times.

A triangles mesh appears. You must convert it to a quad mesh:

  \item Convert the triangle mesh to a quad mesh: \\
	\textsf{$<$QUADRANGULATE$>$ $<$ALL$>$} \\
   You can smooth the mesh with:
	\textsf{$<$REGULARIZE$>$ *$<$ALL$>$}	 \\

The final mesh is displayed. 
If there remain some triangles come back to the previous step
and figure out how to modify the points per edge to help
the mesher. Some experience is needed here.

  \item Renumber the mesh, in order to optimize computations: \\
	\textsf{*$<$RENUMBER$>$}

  \item Define the boundary condition for the 4 corner nodes of the model:
   (these nodes belong to 2 external boundaries so they were given
   a reference number =0) \\
        \textsf{(MODIF\_REF) \$n=\$ (corner) \{click close to corner, inside element\}} \\
   Where n is the reference number of one of the 2 boundaries containing
   the corner node. Zooming can be useful.
   The same operation must be performed for the corner nodes of the 
   subdomains belonging to an external boundary, and for the
   the crack tip nodes. \emph{However, as a special case, crack tip nodes must 
   be assigned the -1 tag.}
 
  \item Export the mesh: \\
	\textsf{$<$SAVE$<$} \\
   Two questions are asked in the bottom prompt:
   \begin{itemize}
    \item Format of the file, you must select: \\
	\textsf{"ftq"}
    \item Prefix name for the file \\
        \textsf{"name"}\\
          The resulting file name will be \texttt{name.ftq}
   \end{itemize}
  
  \end{enumerate}

\end{enumerate}


\subsection{Further tips}
\begin{itemize}
\item Whenever possible it is better to mesh a domain with a \emph{structured}
   mesh (a deformed cartesian grid). This can be done with \textsf{(QUADRANGULATE)},
   during the PREPARE step. 
   See our FAQ for further details.
%   Check the EMC2 user's guide for a description of this feature.
%   FURTHER DOC IS NEEDED HERE.

\item To load an existing project, in the construction tool
   or in the preparation mesh tool:\\
	\textsf{$<$RESTORE$<$ "name"}\\ 
   EMC2 will look for the file \texttt{name.emc2\_bd}.
   Beware: the project loaded will replace the actual 
   project if any, there is no superposition.

\item BUG WARNING (13/07/01): the Sun release of EMC2 has a bug
   with the reference indices in the ftq format
   This bug is fixed in the 2.12c version.
   If you work on a Sun station, download the most recent version of the sources,
   rather than the executable, and compile it yourself.

\item To densify (h-refinement) an existing mesh use the script 
   \texttt{SEM2DPACK/POST/href.csh}.
   It edits the \texttt{*.emc2\_bd} file. 
   You can then restore it in EMC2 and save it in \texttt{*.ftq} format.
%   That is how the example \texttt{SEM2DPACK/EMC2SEM/ex2} was obtained 
%   from \texttt{SEM2DPACK/POST/ex1}. 

\item To create a fault, in \textsf{EDIT\_MESH} mode:
  \begin{enumerate}[a.]
    \item Crack an existing edge:\\
	\textsf{(CRACK) (segment)}
    \item Give a reference number to each side of the fault :\\
	\textsf{(MODIF\_REF) \$n=\$ (segment)}
    \item Give the tag "-1" to crack tip nodes:\\
        \textsf{(MODIF\_REF) \$-1=\$ (corner) *\{click close to crack tip node, inside element\}} \\
  \end{enumerate}

\item Note that only Q4 elements (4 control nodes) are supported. For a smoother
description of boundaries Q9 would be desirable.

\section{Generating a mesh with CUBIT}

An example CUBIT "journal file" is provided in \texttt{POST/mesh2d/dipping\_fault\_2d.jou}.
It exports a mesh file \texttt{dipping\_fault\_2d.inp} in ABAQUS' INP format.
This file must then be translated into SEM2DPACK's MESH2D format following the 
instructions in the next section.

\section{Importing a mesh from CUBIT, EZ4U or other mesh generation software}

The most convenient way of doing this is by writing 
a Matlab function that reads the external mesh file,
performs mesh manipulations if required
and exports a mesh file with the MESH2D format introduced in \secref{mesh2d}.
Two examples of mesh file reading functions are provided in \texttt{POST/mesh2d}.
\texttt{READ\_DCM} reads a mesh in the DCM format of EZ4U (http://www-lacan.upc.es/ez4u.htm).
\texttt{READ\_INP} reads a mesh in the INP format of ABAQUS exported by CUBIT.
The mesh is exported from Matlab in MESH2D format by the functionq \texttt{MESH2D\_WRITE}.

\end{itemize}

\chapter{The solver SEM2D}
\label{Cha:sem2d}

\section{About the method}
\label{Sec:semethod}

Given a crustal model meshed with quadrilateral elements
and a set of material properties, sources, receivers and boundary conditions,
SEM2D solves the elastic wave equation applying 
the Spectral Element Method (SEM) for the space discretization
and a second-order explicit scheme for the time discretization.
The range of physical problems solved by SEM2D
(material constitutive equations and boundary conditions)
is described in more detail in \charef{phys}.
The SEM, introduced by \citeN{Pat84} in Computational Fluid Dynamics,
can be seen as a domain decomposition version of the Spectral Method
or as a high order version of the Finite Element Method.
It inherits from its parent methods the accuracy (spectral 
convergence), the geometrical flexibility 
and the natural implementation of mixed boundary conditions.

Introductory texts to the SEM can be found 
at \url{www.math.lsa.umich.edu/~karni/m501/boyd.pdf} 
(chapter draft, by J.P. Boyd), at
\url{www.mate.tue.nl/people/vosse/docs/vosse96b.pdf}
(a tutorial exposition of the SEM and its connection
to other methods, by F.N. van de Vosse and P.D. Minev)
and at \url{www.siam.org/siamnews/01-04/spectral.pdf} (a perspective paper).
Details about the elastodynamic algorithm and study of some of its properties
are presented by \shortciteN{Kom97}, \shortciteN{KoVi98}, \shortciteN{KoVi99},
\citeN{KomTro99} and \shortciteN{VaCaSaKoVi99}.

The implementation of fault dynamics is similar to that in FEM
with the ``traction at split nodes" method explained by \citeN{And99}.
More details can be found in the author's 
Ph.D. dissertation \cite{Amp02}\footnote{
\url{web.gps.caltech.edu/~ampuero/publications.html}},
in Gaetano Festa's Ph.D. dissertation\footnote{\url{people.na.infn.it/~festa/}}
and in \shortciteN{KanLapAmp08}.

A more accesible tutorial code, SBIEMLAB written in Matlab, 
can be downloaded from the author's website, at
\url{web.gps.caltech.edu/~ampuero/software.html}.

\section{Basic usage flow}

In general, a simulation requires the following steps:
\begin{senumerate}
  \item Prepare the input file \texttt{Par.inp} (\secref{input} and \secref{inblo}).
  \item Run the solver in ``check mode'' (\texttt{iexec=0} in the \texttt{GENERAL} input block of \texttt{Par.inp}):
        \texttt{sem2dsolve > info \&}.
  \item Verify the resolution, stability, estimated CPU cost and memory cost (\secref{check}).
  \item If needed go back to step 1 and modify \texttt{Par.inp} (\secref{check}), else proceed to next step.
  \item Run the solver in ``production mode'' (\texttt{iexec=1}): \texttt{sem2dsolve}.
  \item Plot and manipulate the solver results (\secref{output}).
\end{senumerate}
Full details are given in the following sections.

\section{General format of the input file}
\label{Sec:input}

The input file must be called \texttt{Par.inp}.
Its typical structure is illustrated by two examples
in \figref{ParInp1} and \figref{ParInp2}.
Most of the file is made of standard FORTRAN 90 NAMELIST input blocks.
Each block gives input for a specific aspect of the simulation:
material properties, sources, receivers, boundary conditions, etc.

The general syntax of a NAMELIST block can be found in any FORTRAN 90 textbook.
In summary, a block named \texttt{STUFF}, with possible input arguments 
\texttt{a}, \texttt{b} and \texttt{c}, must be given as
\begin{verbatim}
 &STUFF a=..., b=..., c=... /
\end{verbatim}
where \texttt{...} are user input values. 
Line breaks and comments preceded by \texttt{!} are allowed within an input block.

The complete Reference Guide of the input blocks is presented in \secref{inblo}.
For each block the documentation includes its name, 
possibly the name of a group of blocks to which it belongs,
its purpose, its syntax, the list of its arguments with their description, and some important notes.
In the syntax description, a vertical bar \texttt{|} between two arguments means ``one or the other''.
In the argument list, each item is followed by two informations within brackets \texttt{[]}.
The first bracketed information is the type of the argument: 
double precision (\texttt{dble}), integer (\texttt{int}),
logical (\texttt{log}), single character (\texttt{char}), 
fixed length word (e.g. \texttt{char*6} is a 6 characters word),
arbitrary length word (\texttt{name}) or vectors (e.g. \texttt{int(2)} is a two element integer vector).
The second bracketed information is the default value of the argument.
Some arguments are optional, or when absent they are automatically assigned the default values.

Some arguments have a second version with a suffix \texttt{H}
that allows to set values that are spatially non uniform.
The \texttt{H}-version of the argument must be set to the name of any of the input blocks
of the \texttt{DISTRIBUTIONS} group.
The appropriate \texttt{\&DIST\_xxxx} block must follow immediately.
For example, to set the argument \texttt{eta} to a Gaussian distribution:
\begin{verbatim}
&MAT_KV etaH='GAUSSIAN' /
&DIST_GAUSSIAN length=1d6,100d0, ampli=0.1d0 /
\end{verbatim}
Arguments that accept an \texttt{H}-version are indicated in \secref{inblo}.
When more than one \texttt{H}-version argument is present,
the \texttt{\&DIST\_xxxx} blocks must appear in the same order as in
the argument list of \secref{inblo}.

In the next section, Input Block Reference Guide,
you should get acquainted with the syntax of the blocks you are most likely to use.
The mandatory or more important input blocks are:

\begin{sitemize}
\item \texttt{\&GENERAL}
\item \texttt{\&MESH\_DEF}, followed by a \texttt{\&MESH\_Method} block
\item \texttt{\&MATERIAL}, followed by a \texttt{\&MAT\_Material} block
\item \texttt{\&BC\_DEF}, one for each boundary condition, each followed by a \texttt{\&BC\_Kind} block
\item \texttt{\&TIME}
\item \texttt{\&SRC\_DEF}, followed by \texttt{\&STF\_SourceTimeFunction} and \texttt{\&SRC\_Mechanism} blocks
\item \texttt{\&REC\_LINE}
\end{sitemize}

\begin{figure}[p]
\ImgC{Par.inp.ex1.ps}{0.95}
\caption{\label{Fig:ParInp1} Input file \texttt{Par.inp} for an
elementary example in \texttt{EXAMPLES/TestSH/}
: a boxed region with a structured mesh.}
\end{figure}

\begin{figure}[p]
\ImgC{Par.inp.ex2.ps}{0.95}
\caption{\label{Fig:ParInp2} Input file \texttt{Par.inp} for a
more realistic example: a sedimentary basin with an unstructured mesh generated
by \texttt{EMC2}. Available in \texttt{EXAMPLES/UsingEMC2/}.}
\end{figure}
 
 \newpage
\section{Input Blocks Reference Guide}
\label{Sec:inblo}
%% vertical spacing
\renewcommand\arraystretch{1.5}% (=1.0 is for standard spacing)

\newenvironment{my_enumerate}{
\begin{enumerate}
  \setlength{\itemsep}{1pt}
  \setlength{\parskip}{0pt}
  \setlength{\parsep}{0pt}}{\end{enumerate}
}

\subsection{Sources}

\subsubsection{SRC\_DEF}
{\it Purpose:} Define the sources.

{\it Syntax:} \texttt{\&SRC\_DEF   TimeFunction,mechanism,coord /}\\
followed by blocks of the groups SRC\_TIMEFUNCTION and SRC\_MECHANISM 

{\it Arguments:}

\begin{tabular}[t]{lllp{0.5\linewidth}}
\texttt{TimeFunction} &  name& none& The name of the source time function: 'RICKER', 'TAB' or 'STF\_USER' \\
\texttt{mechanism} & name & none & The name of the source mechanism: 'FORCE', 'EXPLOSION', 'DOUBLE\_COUPLE', 'MOMENT' or 'WAVE' \\
\texttt{coord}  & dble & huge & Location of the source (m). \\
\texttt{file}  & string & 'none' & 
% \begin{minipage}[t]{\linewidth}
Station coordinates and delay times can
be read from an ASCII file, with 3 columns per line:
\begin{my_enumerate}
  \item X position (in m),
  \item Z position (in m) and
  \item delay (in seconds) 
\end{my_enumerate}
that's it.
%\end{minipage}
\end{tabular}

{\it Notes:}
\begin{enumerate}
  \item bla bla 
  \item bla bla
\end{enumerate}


\input{selfdoc.tex}
 \newpage

\section{Verifying the settings and running a simulation}
\label{Sec:check}
 
Once the code has been successfully compiled, the simulation
can be started by typing \texttt{sem2dsolve} from your working directory, 
which contains the file \texttt{Par.inp}.
The computations can be run in background
and the screen output saved in a file (e.g. \texttt{info}) by typing
\texttt{sem2dsolve > info \&}.

A typical screen output of SEM2D, corresponding
to the first example, is shown on the following pages.
The parameters of the simulation 
and some verification information are reported there in a self-explanatory form. 
You are advised to
do a first run with \texttt{iexec=0} in the \texttt{GENERAL} input block
and check all these informations prior to the real simulation.
You should always verify the following:

\begin{itemize}

\item {\bf Stability:}
the CFL stability number should be smaller than $0.55 \sim 0.60$ for second order time schemes,
and much smaller for highly deformed meshes (see Section \ref{faq_sem2d} on
``Instabilities in very distorted elements'').
This number is defined at each computational node as $$\mbox{CFL} = c_P\ \Delta t /\Delta x$$
where $\Delta t$ is the timestep, 
$c_P$ the P-wave velocity and $\Delta x$ the local grid spacing. 
Note that $\Delta x$ is usually much smaller than the element size $h$
($\approx$ \texttt{Ngll}$^2$ times smaller)
because SEM internally subdivides each element onto a non-regular
grid of \texttt{Ngll$\times$Ngll} nodes clustered near the element edges
(Gauss-Lobatto-Legendre nodes).
If the computation is unstable, the maximum displacement,
printed every \texttt{ItInfo} time steps, increases exponentially with time.
Stability can be controlled by decreasing \texttt{Dt} 
or \texttt{Courant} in \texttt{Par.inp}.

\item {\bf Resolution:}
the number of nodes per shortest wavelength $\lambda_{min}$ should be larger than $4.5 \sim 5$.
The minimum wavelength is defined as
$$\lambda_{min}=\min(c_S)/f_{max}$$ 
where $c_S$ is the S-wave velocity and 
$f_{max}$ the highest frequency you would like to resolve, 
e.g. the maximum frequency at which the source spectrum has significant power
(for a Ricker wavelet $f_{max} = 2.5\times f_0$).
For an element of size $h$ 
and polynomial order $p=\texttt{Ngll}-1$, the number of nodes per wavelength $G$
is $$G = \frac{p\,\lambda_{min}}{h}.$$
Typical symptoms of poor resolution are ringing and dispersion of the higher frequencies.
However, in heterogeneous media
these spurious effects might be hard to distinguish from a physically complex wavefield,
so mesh resolution must be checked beforehand. 
If resolution is too low the mesh might be refined by 
increasing \texttt{Ngll} in \texttt{Par.inp} ($p$-refinement) 
or by generating a denser mesh ($h$-refinement).
If you were using EMC2 as a mesh generator,
the script \texttt{PRE/href.csh} can be useful for $h$-refinement.

\item {\bf Cost:}
the total CPU time an memory 
required for the simulation are as much as you can afford.
Estimates of total CPU time are printed at the end of check mode.
Details about memory usage can be found in \texttt{MemoryInfo\_sem2d.txt}.

\end{itemize}

\begin{figure}
\ImgC{Stab_Reso_ex.eps}{1}
\caption{\label{Fig:stabres} Checking the quality of a mesh
with \texttt{PRE/ViewMeshQuality.m} for the example in \texttt{EXAMPLES/UsingEMC2/}.
The balance of the stability and resolution properties of the mesh can be analyzed:
logarithmic stability index (top) and logarithmic resolution index (bottom).
Histograms of these indices (in number of elements) are shown on the right.}
\end{figure}

The quality of the mesh can be inspected with
the Matlab script \texttt{PRE/ViewMeshQuality.m} which produces
plots like Figure \ref{Fig:stabres}.
The proper balance of the mesh with respect to the following
two criteria can be analyzed:

\begin{itemize}

\item {\bf Stability criterion}, related to the largest stable timestep. 
The stability of each element is quantified by
$$ S = \min(\Delta x/c_P). $$
We also define a stability index as 
$$ SI = \log[ S / \text{median}(S) ].$$
where the median value is taken over the whole mesh.
Red elements (small SI) are relatively unstable
and require small timesteps $\Delta t$.
Because $\Delta t$ is constant over the whole mesh
and the computational cost is inversely proportional to $\Delta t$,
these red elements penalize the computational efficiency.
The mesh should be redesigned to increase their size, as much as possible,
while keeping them small enough to resolve the shortest wavelength (see next).
%Conversely, elements with very high SI (blue) could in principle be smaller.

\item {\bf Resolution criterion},
related to the number of nodes per shortest wavelength. 
The resolution of each element is quantified by
$$ R = \min(c_S / h). $$
We also define a resolution index as 
$$ RI = \log[ R / \text{median}(R) ].$$
where the median value is taken over the whole mesh.
Red elements (small RI) have relatively poor resolution,
in their vicinity the maximum frequency resolvable by the mesh is limited.
The mesh should be redesigned to decrease their size, as much as possible.
Conversely, elements with very high RI (blue) are smaller than required
and might increase the computational cost.

\end{itemize}

To minimize the CPU and memory cost of a simulation
an ideal mesh design should minimize the spread of the two indices above,
by aiming at a ratio of element size to wave velocity, $h/c$, 
as uniform as possible across the whole mesh.
%Resolution and stability are sometimes competing constraints in mesh design,
%trade-offs must be resolved through experience.
%To improve the quality of your mesh 
%try to use larger elements in the regions where very low stability was observed
%but keep them large enough to resolve the shortest wavelength.
However, in some cases a poorly balanced mesh is inevitable: in the example 
of Figure \ref{Fig:stabres} the worst elements are near the edges of the
sedimentary basin, at a sharp velocity contrast.
Small element sizes on the rock side are inherited from the sediment mesh.\footnote{In 
future releases of SEM2DPACK
this penalty on computational efficiency will be reduced by 
non-conformal meshing with mortar elements,
by timestep subcycling or by implicit/explicit timestep partitioning.}

Similar information is plotted by
\texttt{gv Stability\_sem2d.ps} and \texttt{gv Resolution\_sem2d.ps}.
The indices in these files are however not logarithmic and are
not normalized by the median.\\\

% include this file in your .tex :
%   % include this file in your .tex :
%   \input{info.ex1.ps.tex}
\ImgC{info.ex1_1.ps}{1}
\ImgC{info.ex1_2.ps}{1}

\ImgC{info.ex1_1.ps}{1}
\ImgC{info.ex1_2.ps}{1}

 
\section{Outputs, their visualization and manipulation}
\label{Sec:output}

In addition to the screen output described above,
\texttt{sem2dsolve} generates different
files and scripts that allow the user
to control the parameters of the simulation and to display the results.
All the outputs files follow the naming convention \texttt{SomeName\_sem2d.xxx},
where \texttt{xxx} is one of the following extensions:
\texttt{tab} for ASCII data files, \texttt{txt} for other text files,
\texttt{dat} for binary data files, etc.
This makes it easy to clean a working directory with a single command like
\texttt{rm -f *\_sem2d*}.

\subsection{Spectral element grid}

As explained in the previous section, 
\texttt{sem2dsolve} generates two PostScript files for 
mesh quality checking purposes:
\texttt{Stability\_sem2d.ps} and \texttt{Resolution\_sem2d.ps}.
The relevant information is contained
in the files \texttt{Stability\_sem2d.tab} and \texttt{Resolution\_sem2d.tab}
and can also be inspected with the Matlab script \texttt{PRE/ViewMeshQuality.m}.

\subsection{Source time function}

\texttt{sem2dsolve} generates a file called
\texttt{SourcesTime\_sem2d.tab} containing
the source time function sampled at the same rate as the receivers.
It is important to verify that the spectrum of the source
has little power at those high frequencies that are not well resolved by the mesh
(those that correspond to less than 5 nodes per wavelength).
If this is not the case you must be very cautious in the interpretation of 
the seismograms in the high frequency range, or low-pass filter the results.

\subsection{Snapshots}

\texttt{sem2dsolve} generates snapshots
at a constant interval defined, in number of solver timesteps, 
by the input parameter \texttt{itd} of the \texttt{SNAP\_DEF} input block.
An example is shown in \figref{snap}.
Requested fields are exported in binary data files called \texttt{xx\_XXX\_sem2d.dat},
where \texttt{xx} is the field code defined in the documentation of the \texttt{PLOTS} input block
and \texttt{XXX} is the 3-digit snapshot number.
The user is encouraged to inspect the Matlab s
function \texttt{POST/sem2d\_snapshot\_read.m}
to find more about the data formats and their manipulation.

Snapshots can also be exported as PostScript files \texttt{xx\_XXX\_sem2d.ps}.
These can be merged into an animated GIF (movie) file \texttt{movie.gif} 
by the script \texttt{POST/movie.csh}
and displayed by \texttt{xanim movie.gif} or \texttt{animate movie.gif}.
An animated GIF can also be created by the 
Matlab function \texttt{POST/sem2d\_snapshot\_movie.m}.

\begin{figure} %[p]
\ImgCL{snapshot.ps}{0.8}
\caption{\label{Fig:snap} Sample snapshot from \texttt{EXAMPLES/UsingEMC2/}:
an obliquely incident SH plane wave impinging on a sedimentary basin. 
The unstructured mesh of spectral elements is plotted on background.}
\end{figure}

\subsection{Seismograms}

The seismograms are stored using the SEP format, a simple
binary block of single precision floats. The components 
of the vector field (velocity by default) are stored
in separate files \texttt{U*\_sem2d.dat},
where \texttt{*} is \texttt{x} or \texttt{z} in P-SV 
and \texttt{y} in SH.
The seismograms header is in the file \texttt{SeisHeader\_sem2d.hdr}.
Its second line contains the sampling timestep \texttt{DT}, the
number of samples \texttt{NSAMP} and the number of stations \texttt{NSTA}.
The stations coordinates, \texttt{XSTA} and \texttt{ZSTA}, are listed
from the third line to the end of file.
With this notations, \texttt{U*\_sem2d.dat} contains a 
\texttt{NSAMP}$\times$\texttt{NSTA} single precision matrix.

You can view the seismograms using any tool
that is able to read the SEP format, which is the case of almost
all the softwares able to deal with seismic data.
\texttt{sem2dsolve} generates scripts for the 
XSU-Seismic Unix visualization tool\footnote{Seismic Unix is freely available from the Colorado 
School of Mines at \url{http://timna.mines.edu/cwpcodes}}:

\begin{sitemize}
\item \texttt{Xline\_sem2d.csh} displays all seismograms together on screen
\item \texttt{PSline\_sem2d.csh} plots all seismograms
on PostScript files \texttt{U*Poly\_sem2d.ps}
\item \texttt{Xtrace\_sem2d.csh} prompts the user for a trace number
(between 1 and \texttt{NSTA})
and then displays this particular trace on screen
\item \texttt{PStrace\_sem2d.csh} does the same as \texttt{Xtrace},
but exports the traces as PostScript files \texttt{U*TraceXXX\_sem2d.ps} 
where \texttt{XXX} is the number of that particular trace
\end{sitemize}

The program \texttt{post\_seis.exe} performs similar
basic manipulation and plotting (through \texttt{gnuplot}) of the 
seismograms. Its interactive menu is self-explanatory.
It is usually called inside a script, as in \texttt{POST/seis\_b2a.csh}
(converts all seismograms to ASCII) or \texttt{POST/seis\_plot.csh}
(plots all seismograms together, an example is shown in \figref{seis}).

The script \texttt{POST/sample\_seis.m} shows how
to manipulate and plot seismogram data in Matlab.
It uses the functions \texttt{POST/sem2d\_read\_seis.m} and \texttt{POST/plot\_seis.m}.

\begin{figure} %[p]
\ImgCL{Uy_sem2d.ps}{0.58}
\caption{\label{Fig:seis} Sample seismograms from \texttt{EXAMPLES/UsingEMC2/}
generated with \texttt{POST/seis\_plot.csh}.}
\end{figure}

\subsection{Fault outputs}

Fault data from dynamic rupture simulations is stored
in three files (where \texttt{XX} is the boundary tag of the first
side of the fault, \texttt{tags(1)} of the \texttt{BC\_SWFFLT} input block):
\begin{sitemize}
  \item \texttt{FltXX\_sem2d.hdr} contains the information needed
to read the other fault data files. Its format, line by line, is:
\begin{enumerate}
\item \texttt{NPTS NDAT NSAMP DELT} (name of parameters) 
\item Value of parameters above
\item Name of fields exported in \texttt{FltXX\_sem2d.dat}, separated by ``:"
\item \texttt{XPTS ZPTS} (name of coordinate axis)
\item from here to the end of file: a two-column table of coordinates
of the output fault nodes
\end{enumerate}

  \item \texttt{FltXX\_sem2d.dat} contains the space-time distribution of fault data such as slip, slip rate, stress and strength. 
Every \texttt{DELT} seconds a block of fault data values is written. The total number of blocks is \texttt{NSAMP}.
Each block has \texttt{NDAT} lines (one per fault data field) 
and \texttt{NPTS} columns (one per fault node)
\footnote{The actual number of columns is \texttt{NPTS} +2:
Fortran adds a one-word tag at the front and end of each record.}. 
Stresses are relative to their initial values.

  \item \texttt{FltXX\_init\_sem2d.tab} contains the spatial distribution of 
initial shear stress, initial normal stress and initial friction (3 columns).

  \item \texttt{FltXX\_potency\_sem2d.tab} contains time-series of seismic potency and potency rate. 
The seismic potency tensor $p_{ij}$ is defined by the following integral along the fault:
\begin{equation}
p_{ij} = \frac{1}{2} \int_{fault} (n_i \Delta u_j + n_j \Delta u_i)  \ dx
\end{equation}
where $\Delta u$ is slip and $n$ is the local unit vector normal to the fault.
The file contains one line per timestep.
In SH (\texttt{ndof=1}) each line has $4$ columns: $2$ components of potency ($p_{13}$ and $p_{23}$)
and $2$ components of potency rate ($\dot{p}_{13}$ and $\dot{p}_{23}$).
In P-SV (\texttt{ndof=2}) each line has $3$ components of potency ($p_{11}$, $p_{22}$ and $p_{12}$)
and $3$ components of potency rate ($\dot{p}_{11}$, $\dot{p}_{22}$ and $\dot{p}_{12}$).
\end{sitemize}

Some tools are available to manipulate the data in \texttt{FltXX\_sem2d.dat}:
\begin{sitemize}
\item The script \texttt{FltXX\_sem2d.csh} shows 
how to extract ASCII time series of different 
fields at given locations on the fault, using Seismic Unix tools.
\item The program \texttt{post\_fault.exe} performs basic manipulations of the 
fault data, including conversion to an ASCII file readable by \texttt{gnuplot}. 
Its interactive menu is self-explanatory.
\item The script \texttt{POST/sample\_fault.m} 
and function \texttt{POST/sem2d\_read\_fault.m} show how
to manipulate and plot fault data in Matlab.
\end{sitemize}

\subsection{Stress glut}

For damage and plastic materials,
the solver can export the plastic and damage components of the
cumulative stress glut tensors 
defined, respectively, as
\eqa
  s^p_{ij}(t) &=& -\iint 2\mu\, \epsilon^{p}_{ij}(t) \ dx\,dz\\
  s^d_{ij}(t) &=& \iint [ \sigma_{ij}(t) - c^0_{ijkl} \epsilon^e_{kl}(t) ]  \ dx\,dz
\ena
where $\epsilon^p$ is the plastic strain, $\epsilon^e$ the elastic strain,
$\sigma$ the absolute stress
and $c^0$ the tensor of elastic moduli of the undamaged medium. 

To enable this feature: set \texttt{COMPUTE\_STRESS\_GLUT = .true.} 
in file \texttt{SRC/constants.f90}, then re-compile the code.
The stress glut output is exported in the file \texttt{stress\_glut\_sem2d.tab}
in 7 columns: time, $s^p_{11}$, $s^p_{22}$, $s^p_{12}$, $s^d_{11}$, $s^d_{22}$, $s^d_{12}$.

\subsection{Energies}

The solver can export the 
cumulative plastic energy, the kinetic energy and the total change of elastic energy,
defined respectively by
\eqa
 E^p(t) &=& \iiint_0^t \sigma_{ij}(t') \dot{\epsilon}^p_{ij}(t') \ dx\,dz\,dt'\\
 E^k(t) &=& 1/2\ \iint \rho v_i^2(t) \ dx\,dz\\
 E^e(t) &=& \iint U[\epsilon^e(t)] - U[\epsilon^e(0)] \ dx\,dz
\ena
where $U$ is the elastic potential and summation over subindices is implied.

To enable this feature: set \texttt{COMPUTE\_ENERGIES = .true.}
in file \texttt{SRC/constants.f90}, then re-compile the code.
The energy output is exported in the file \texttt{energy\_sem2d.tab} 
in 4 columns: time, $E^p$, $E^k$ and $E_e$.

\subsection{Matlab utilities}
\label{Sec:matlab}

A range of functions and sample scripts for Matlab are available 
to read, manipulate and plot output data.
Add the directory \texttt{POST/} to your Matlab path (\texttt{addpath}).
For an overview of existing utilities, type \texttt{help POST}:

\begin{verbatim}
   SEM2DPACK/POST provides Matlab utilities for the manipulation
   and visualization of SEM2DPACK simulations results.

   Reading simulation data:

    SEM2D_READ_SPECGRID reads a spectral element grid
    SEM2D_SNAPSHOT_READ reads snapshot data
    SEM2D_READ_SEIS     reads seismogram data
    SEM2D_READ_FAULT    reads fault data

   Data manipulation:

    SEM2D_EXTRACT_POINT extracts field values at an arbitrary point
    SEM2D_EXTRACT_LINE  extracts field values along a vertical or horizontal line
    ARIAS_INTENSITY     computes Arias Intensity and Significant Duration
    RESPONSE_SPECTRUM   computes response spectra (peak dynamic response
                        of single-degree-of-freedom systems)

   Data visualization:

    SEM2D_PLOT_GRID     plots a spectral element grid
    SEM2D_SNAPSHOT_PLOT plots snapshot data
    SEM2D_SNAPSHOT_GUI  interactively plots snapshot data
    SEM2D_SNAPSHOT_MOVIE makes an animation of snapshot data
    PLOT_MODEL          plots velocity and density model
    PLOT_SEIS           plots multiple seismograms
    PLOT_FRONTS         space-time plot of rupture front and process zone tail
    SAMPLE_FAULT        example of visualization of fault data
    SAMPLE_SEIS         example of visualization of seismogram data

   Miscellaneous tools:

    XCORRSHIFT          cross-correlation time-delay measurement
    SPECSHIFT           signal time shift by non-integer lag via spectral domain
    SPECFILTER          zero-phase Butterworth filter via spectral domain

\end{verbatim}

\chapter{Adding features to SEM2D (notes for advanced users)}
\label{Cha:add}

Sometimes you will need to add new capabilities to the
SEM2DPACK solver, by modifying the program.
The following notes are intended to guide you through this process.
We will not give here a comprehensive description of the code architecture,
only enough details to get you started in performing safely
the most usual and evident modifications.

\section{Overview of the code architecture}

{\it [ ... in progresss ...]}

This code uses a mixture of procedural (imperative) and object-oriented paradigms.
Historically, it evolved from a purely procedural code.

Extensive use of modularity.

Object Oriented Programming (OOP) features (principles) applied in this code:
encapsulation, classes, %or abstract objects), 
static polymorphism. 
%inheritance (in mesh generation modules?, "has-a" construct but not "is-a"), 
These are not applied everywhere in the code, for different reasons:
reusage of legacy code, performance, 
%ex: implementing the non anticipation principle (Zimmerman et al 1992, Duboi-Pelerin etal 1992) has large overhead
difficulty related to the limits of Fortra 90, 
or sections of code yet to be updated. 
% Missing in F90: dynamic polymorphism, full inheritance, templates. 

Added cost of structures containing pointer components: 
the possibility of pointer aliasing prevents more agressive compiler optimizations
and adds overhead for safety checks.
% Zimmerman et al 1992: FEM OO

\section{Accessible areas of the code}

Some areas of the code have been written in such a way that a moderately 
experienced Fortran 95 programmer,
with a limited understanding of the code architecture,
can introduce new features without breaking the whole system.
This is achieved through modularity, encapsulation and templates.
The modifications that are currently accessible are:
\begin{itemize}
  \item boundary conditions, see \texttt{bc\_gen.f90}
  \item material rheology, see \texttt{mat\_gen.f90}
%  \item friction laws
  \item source time functions, see \texttt{stf\_gen.f90}
  \item spatial distributions, see \texttt{distribution\_general.f90}
%   \item output fields
\end{itemize}
The source files listed above contain step-by-step instructions,
just follow the comments starting by \texttt{!!}.


\chapter{Frequently Asked Questions}
\label{Cha:faq}

\section{SEM2D}
\label{faq_sem2d}

\subsection*{Segmentation fault}

This problem is often related to a small stack size in your computer settings.
In your Linux shell do: \texttt{ulimit -s unlimited} under bash
or \texttt{limit stacksize unlimited} under csh.
Place this command in your startup files (\texttt{.login}, \texttt{.bashrc} or \texttt{.cshrc}).

\subsection*{Instabilities on very distorted elements}

Very distorted elements (with very small or very large angles) 
are usual close to wedges of sedimentary basins, fault branching points, etc.
In general, distorted elements are less stable than square elements:
spurious motions with exponentially increasing amplitude might appear in their vicinity.
In most cases these instabilities can be suppressed by reducing the \texttt{Courant} input parameter.
There is currently no simple recipe to determine the maximum value of this parameter,
so trial and error is required.

\section{EMC2}
\label{faq_emc2}

\subsection*{I can't get rid of a few triangles}

Obtaining a quality quad mesh is not always a trivial task. Trial and error and experience
is needed. This can be by far the most time consuming stage of modeling.

First make sure that the total number of element edges
along the perimeter of each mesh domain is even. This is a necessary
topological condition to generate a quad-only mesh. 

When the geometry seems too complicated for quad meshing you
should consider simplifying the geometry, especially those details that are much smaller
than the dominant wavelength.

If the above fails or does not apply, you have to help the mesher. 
The recommended procedure in EMC2 is:

\begin{enumerate}
\item Divide your original mesh into simple domains,
in such a way that \emph{most} domains have exactly four sides (possibly curved)
and the remaining non-four-sided domains are as small as possible.
\item Generate a structured quad-mesh (a regular grid) inside each four-sided
domain with the \textsf{(QUADRANGULATE)} tool of the \textsf{PREP\_MESH} mode,
as described in section 5.2.13 of EMC2's manual
(note that this is \emph{not} the same as the \textsf{$<$QUADRANGULATE$>$} 
button in the \textsf{EDIT\_MESH} mode). 
\item Proceed as usual (triangulation followed by quadrangulation) 
inside the remaining non-four-sided domains. 
If these are small enough EMC2 should not have problems doing a
correct tri-to-quad meshing.
\end{enumerate}


\bibliographystyle{chicagoa}
\bibliography{Biblio1}

\end{document}
