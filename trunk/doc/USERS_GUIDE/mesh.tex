\chapter{Mesh generation}
\label{Cha:meshing}

\section{General guidelines}

The Spectral Element Method (SEM) requires an initial decomposition 
of the space domain into quadrilateral elements (a quad mesh).
Obtaining the best performance (accuracy/cost) out of the SEM imposes constraints
on the mesh design:
\begin{sitemize}
\item The interfaces between different materials, 
at which sharp contrasts of material properties occur,
should {\it preferably} coincide with faces of the elements. 
This is sometimes called an {\it adapted mesh} 
and is the only way to preserve spectral accuracy at material interfaces.
\item Fault planes, across which displacement discontinuities occur,
{\it must} coincide with element faces. 
Faults are implemented with a {\it split node} formulation.
\item Elements can be deformed, but extremely small and extremely large 
angles between faces of a same element must be avoided. 
This would penalize both accuracy and stability.
\item The linear size of the elements must be small enough, so that each element
contains enough computational nodes per minimum wavelength,
and each fault boundary element contains enough nodes per rupture process zone.
\item Unnecessarily small elements should be avoided, 
they penalize the stability of the method. 
\end{sitemize}
Generating high quality quad meshes for complicated geological models
is not yet a fully automated process, and can be very time-consuming.
Iterations between mesh generation and mesh quality check are sometimes required.
The last two constraints above are addressed more quantitatively in \secref{check}.
Mesh quality assessment tools are also presented in \secref{check}.

The remainder of this chapter describes three possible ways to generate quad meshes,
by order of complexity:
\begin{senumerate}
\item If the geometrical structure of the model is simple
or if the user prefers to sacrifice accuracy by using a non-adapted structured mesh,
i.e. a logically cartesian mesh 
where the element faces do not necessarily follow the material interfaces,
the basic built-in meshing capabilities of the solver SEM2D, described
in \secref{mesh_inc}, are sufficient.
\item Moderately complicated meshes can be generated with the included Matlab tools,
described in \secref{mesh2d}.
\item Adapted meshes for more complicated geological models 
must be generated with some external software.
As an illustration, the usage of the freely available 2D mesh generation software EMC2 
is described in \secref{emc2}.
\end{senumerate}

SEM2DPACK provides only basic meshing capabilities
and does not include an unstructured mesh generator 
for complicated, realistic geological models. 
This chapter describes how to achieve that task with an external software, EMC2.

Generating a high quality unstructured quad mesh can be a time-consuming task.
Let's note that, for wave propagation problems without dynamic faults, 
if the acceptable accuracy is low
(or large computational resources are available to work with a very fine mesh)
a structured mesh in which the element faces do not necessarily follow the material interfaces
can be generated with the basic built-in meshing capabilities of SEM2D.

\section{Meshing features included in the solver}
\label{Sec:mesh_inc}

The solver itself has very limited meshing capabilities.
It can only generate a structured mesh for a single quadrilateral domain,
possibly with curved sub-horizontal boundaries and curved sub-horizontal layer interfaces.
The domain can be cut in the horizontal direction by a single fault, possibly curved or kinked.

For further details see the Reference Guide for the input blocks
\texttt{MESH\_CART} and  \texttt{MESH\_LAYERED} in \secref{inblo}.

\section{Meshing with the MESH2D Matlab utilities}
\label{Sec:mesh2d}

A number of Matlab functions for 2D meshing are provided in \texttt{POST/mesh2d}.
These can generate structured meshes for quadrilateral domains
with curved boundaries, and merge several such meshes to generate
a more complicated, globally unstructured mesh.
Functions for manipulating, visualizing and exporting these meshes are included.
Here is an overview of available tools:

\begin{verbatim}
  SEM2DPACK/PRE/mesh2d provides Matlab utilities for the generation, manipulation 
  and visualization of structured 2D quadrilateral meshes, and unstructured 
  compositions of structured meshes.
 
  The mesh database is stored in a structure described in MESH2D_TFI's help.
 
  Mesh generation:
 
   MESH2D_TFI     	generates a structured mesh by transfinite interpolation
   MESH2D_QUAD    	generates a structured mesh for a quadrilateral domain
   MESH2D_CIRC_HOLE	generates a mesh for a square domain with a circular hole 
   MESH2D_WEDGE		generates a mesh for a triangular wedge domain
   MESH2D_EX0		mesh for a vertical fault
   MESH2D_EX1		mesh for a shallow layer over half-space with dipping fault
 
  Mesh manipulation:
 
   MESH2D_ROTATE   	rotates the node coordinates 
   MESH2D_TRANSLATE	translates the node coordinates
   MESH2D_MERGE    	merges several meshes into a single mesh
   MESH2D_WRITE  	writes a 2D mesh database file (*.mesh2d)
   MESH2D_READ 		reads a 2D mesh database from a *.mesh2d file 
 
  Reading mesh files from other mesh generation software:
 
   READ_DCM 		reads a 2D mesh in the DCM format of EZ4U (http://www-lacan.upc.es/ez4u.htm)
   READ_INP		reads a 2D mesh in the INP format of ABAQUS exported by CUBIT
 
  Mesh visualization:
 
   MESH2D_PLOT        	plots a 2D mesh
 
  Miscellaneous tools:
 
   SAMPLE_SEGMENTS	generates points that regularly sample multiple segments of a line
\end{verbatim}

The functions \texttt{MESH2D\_TFI} and \texttt{MESH2D\_MERGE} are the core tools.
The script \texttt{MESH2D\_EXAMPLE1} is a good starting point.
The syntax of the mesh database file, \texttt{*.mesh2d}, is described
in \secref{inblo}.

\section{Generating a mesh with EMC2}
\label{Sec:emc2}

\subsection{The mesh generator EMC2}

EMC2 is one of the few public domain 2D mesh generation softwares 
that includes quadrilateral elements and a Graphical User Interface. 
Its C code sources and executables can be freely downloaded from\\
%\url{http://www-c.inria.fr/gamma/cdrom/www/emc2/eng.htm}.
\centerline{\url{http://www.ann.jussieu.fr/~hecht/ftp/emc2/}.}\\
We show here an example featuring the most useful functionalities of EMC2.
For further details you should refer to the complete documentation of
the EMC2 package,

Before starting you must prepare files containing in 2-column data 
the coordinates (X,Z)
of all the points needed to define the geometry of the model 
(topography, sediment bottom). 
%An example is given in \texttt{SEM2DPACK/EMC2SEM/palosgrandes.dat}.

Once installed, you can run EMC2 by typing \texttt{emc2}.

\subsection{Notations}

The following notations are assumed in the next section:
\begin{itemize}
\item  \textsf{(XXX)} = click XXX on top menu bar 
\item  \textsf{(xxx)} = click xxx on bottom menu bar
\item  \textsf{$<$XXX$<$} = click XXX on left menu bar
\item  \textsf{$>$XXX$>$} = click XXX on right menu bar
\item  \textsf{\$xxx\$} = enter xxx from keyboard or from the calculator in the right panel
\item  \textsf{"xxx"} = type xxx in bottom prompt
\item  \textsf{\{xxx\}} = perform action xxx
\item  \textsf{*xxx}  = do xxx as many times as needed
\item  \textsf{n*xxx} = do xxx n times
\end{itemize}

\subsection{Basic step-by-step}

A typical EMC2 session has three steps:

\begin{enumerate}[STEP I:]
\item CONSTRUCT\label{cons}, defines the geometry of the model

\begin{enumerate}[1.]
\item Switch to the construction tool:\\
      \textsf{$<$CONSTRUCTION$<$}
\item Load the points:\\
      \textsf{(POINT) (xy file) "palosgrandes.dat"}\\
   You must give the full path to your points-file,
   the root directory being the one where you launched \texttt{emc2}.
   \label{loadpoints} 

\item Reset the figure window to fit all points: \\
	\textsf{$>$SHOW ALL$>$}\\

The original data has some geometrical features 
that are too complex to be meshed by quadrilaterals, for instance
the corners at the N and S ends of the basin, you may
want to smooth out these features. 
You also need to define the extreme 
boundaries of the region to be modelled (N,S and bottom
absorbing boundaries) and some additional points on
the free surface outside the basin.
You must modify the data set (add and delete points):

\item Add new points:

  \begin{enumerate}[a.]
  \item with the mouse:\\
	\textsf{(POINT) (mouse) *\{click in figure window\}}

  \item by coordinates:\\
	\textsf{(POINT) (xy pt) *\{ \$x=y=\$ \}}\\
     This is the safest way to get really vertical and
     horizontal boundaries needed for the absorbing conditions
     in SPECFEM90. You probably need to get
     the coordinates of an existing reference point:\\
        \textsf{(POINT) $<$QUERY$<$ (point) *\{click on point\}}

  \item you can also reload another point-file (\ref{loadpoints})

  \end{enumerate}

\item Delete points, \\
	\textsf{(POINT) $<$DESTRUCT$<$ (point) *\{click on point\}}\\

Now you must define the geometry of the domains. These
macro-blocks are intended to be internally meshed by 
deformed quadrilaterals. Their geometry follows the geometry
of the geological model (one domain per material).
Each domain must be bounded by segments or splines:

\item Segments:\\
	\textsf{(SEGMENT) (point) 2*\{click extreme point\}}

\item Splines:\\
        \textsf{(SPLINE) (point) *\{click point\}}\\
   You will see the spline evolve as you click points.

\end{enumerate}


\item PREPARE, defines the properties of the discrete spectral element mesh

  \begin{enumerate}[1.]

  \item Switch to the preparing mesh tool:\\
	\textsf{$<$PREP MESH$<$}

  \item Define domains with rock n:\\
	\textsf{(DOMAIN REF) \$n=\$ (any) *\{click inside domain\}}\\
   You will see the domains edges get colored and the 
   domains get numbered with n.

  \item At any moment you can decide to show or not the
domain decomposition:\\ 
  To hide the domain decomposition:\\
        \textsf{$>$REFRESH$>$}\\
  Show the domain decomposition:\\
	\textsf{(SHOW) (ALL)}

  \item Remove a domain definition:\\
	\textsf{(REMOVE) (DOMAINE) (any) \{click inside domain\}}\\
   WARNING: corrections to the domain decomposition are
   sometimes displayed only after refreshing the figure window.

  \item Now you must define the subdivision of each 
   domain in quadrilateral finite elements. 
   Define the number $n$ of elements on each edge:\\
	\textsf{(NB INTERVAL) \$n=\$ (any) \{click edge\} }\\
   You will see the intermediate points appear.
   The number of intervals $n$ is mainly dictated by the resolution criterion: 
   elements should be smaller than the smallest wavelength you want to propagate.
   Moreover, {\bf a domain can be quadrangulated only if
   the total number of intervals along its perimeter is even} 
   (the sum of all $n$ along its boundaries).
   However, a quality mesh is not always guaranteed
   and you need to proceed by trial and error (emc2 allows you
   to jump back and forth between the different steps of the
   meshing procedure).

\item Finally you must define the external boundaries of 
the modelled region which will have a special treatment.
You must associate a tag (a number) to each absorbing boundary.
No convention is assumed but you should remember those tags later
when setting the boundary conditions in SEM2D. It is also useful
to assign a tag to the free surface boundary, that will be eventually
used by SEM2D to locate the receivers or sources.

   Define a boundary with index n:\\
	\textsf{(LINE REF) \$n=\$ (any) *\{click edge\}}\\ 
   Of course each boundary can be composed of many 
   domain edges.
   Refresh the display to better see the boundaries.
   \emph{The same procedure applies to define split-node interfaces such as faults and cracks: 
   you must assign a different tag to each side of the fault.}

  \item Save your work in EMC2 format: \\
	\textsf{$<$SAVE$<$ "name"}\\
   The resulting file is \texttt{name.emc2\_bd}
   
  \end{enumerate}

\item EDIT, generates the mesh

  \begin{enumerate}[1.]

  \item Switch to the edit mesh tool: \\
	\textsf{$<$EDIT\_MESH$<$} \\
   Press ENTER 4 times.

A triangles mesh appears. You must convert it to a quad mesh:

  \item Convert the triangle mesh to a quad mesh: \\
	\textsf{$<$QUADRANGULATE$>$ $<$ALL$>$} \\
   You can smooth the mesh with:
	\textsf{$<$REGULARIZE$>$ *$<$ALL$>$}	 \\

The final mesh is displayed. 
If there remain some triangles come back to the previous step
and figure out how to modify the points per edge to help
the mesher. Some experience is needed here.

  \item Renumber the mesh, in order to optimize computations: \\
	\textsf{*$<$RENUMBER$>$}

  \item Define the boundary condition for the 4 corner nodes of the model:
   (these nodes belong to 2 external boundaries so they were given
   a reference number =0) \\
        \textsf{(MODIF\_REF) \$n=\$ (corner) \{click close to corner, inside element\}} \\
   Where n is the reference number of one of the 2 boundaries containing
   the corner node. Zooming can be useful.
   The same operation must be performed for the corner nodes of the 
   subdomains belonging to an external boundary, and for the
   the crack tip nodes. \emph{However, as a special case, crack tip nodes must 
   be assigned the -1 tag.}
 
  \item Export the mesh: \\
	\textsf{$<$SAVE$<$} \\
   Two questions are asked in the bottom prompt:
   \begin{itemize}
    \item Format of the file, you must select: \\
	\textsf{"ftq"}
    \item Prefix name for the file \\
        \textsf{"name"}\\
          The resulting file name will be \texttt{name.ftq}
   \end{itemize}
  
  \end{enumerate}

\end{enumerate}


\subsection{Further tips}
\begin{itemize}
\item Whenever possible it is better to mesh a domain with a \emph{structured}
   mesh (a deformed cartesian grid). This can be done with \textsf{(QUADRANGULATE)},
   during the PREPARE step. 
   See our FAQ for further details.
%   Check the EMC2 user's guide for a description of this feature.
%   FURTHER DOC IS NEEDED HERE.

\item To load an existing project, in the construction tool
   or in the preparation mesh tool:\\
	\textsf{$<$RESTORE$<$ "name"}\\ 
   EMC2 will look for the file \texttt{name.emc2\_bd}.
   Beware: the project loaded will replace the actual 
   project if any, there is no superposition.

\item BUG WARNING (13/07/01): the Sun release of EMC2 has a bug
   with the reference indices in the ftq format
   This bug is fixed in the 2.12c version.
   If you work on a Sun station, download the most recent version of the sources,
   rather than the executable, and compile it yourself.

\item To densify (h-refinement) an existing mesh use the script 
   \texttt{SEM2DPACK/POST/href.csh}.
   It edits the \texttt{*.emc2\_bd} file. 
   You can then restore it in EMC2 and save it in \texttt{*.ftq} format.
%   That is how the example \texttt{SEM2DPACK/EMC2SEM/ex2} was obtained 
%   from \texttt{SEM2DPACK/POST/ex1}. 

\item To create a fault, in \textsf{EDIT\_MESH} mode:
  \begin{enumerate}[a.]
    \item Crack an existing edge:\\
	\textsf{(CRACK) (segment)}
    \item Give a reference number to each side of the fault :\\
	\textsf{(MODIF\_REF) \$n=\$ (segment)}
    \item Give the tag "-1" to crack tip nodes:\\
        \textsf{(MODIF\_REF) \$-1=\$ (corner) *\{click close to crack tip node, inside element\}} \\
  \end{enumerate}

\item Note that only Q4 elements (4 control nodes) are supported. For a smoother
description of boundaries Q9 would be desirable.

\section{Importing a mesh from CUBIT, EZ4U or other mesh generation software}

The most convenient way of doing this is by writing 
a Matlab function that reads the external mesh file,
performs mesh manipulations if required
and exports a mesh file with the MESH2D format \secref{mesh2d}.

Two examples of this are provided in \texttt{POST/mesh2d}.
The functions \texttt{READ\_DCM} and \texttt{READ\_INP}		
read a mesh in, respectively, the DCM format of EZ4U (http://www-lacan.upc.es/ez4u.htm)
and in the INP format of ABAQUS exported by CUBIT.

\end{itemize}
