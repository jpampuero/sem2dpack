\chapter{Adding features to SEM2D (notes for advanced users)}
\label{Cha:add}

Sometimes you will need to add new capabilities to the
SEM2DPACK solver, by modifying the program.
The following notes are intended to guide you through this process.
We will not give here a comprehensive description of the code architecture,
only enough details to get you started in performing safely
the most usual and evident modifications.

\section{Overview of the code architecture}

{\it [ ... in progresss ...]}

This code uses a mixture of procedural (imperative) and object-oriented paradigms.
Historically, it evolved from a purely procedural code.

Extensive use of modularity.

Object Oriented Programming (OOP) features (principles) applied in this code:
encapsulation, classes, %or abstract objects), 
static polymorphism. 
%inheritance (in mesh generation modules?, "has-a" construct but not "is-a"), 
These are not applied everywhere in the code, for different reasons:
reusage of legacy code, performance, 
%ex: implementing the non anticipation principle (Zimmerman et al 1992, Duboi-Pelerin etal 1992) has large overhead
difficulty related to the limits of Fortra 90, 
or sections of code yet to be updated. 
% Missing in F90: dynamic polymorphism, full inheritance, templates. 

Added cost of structures containing pointer components: 
the possibility of pointer aliasing prevents more agressive compiler optimizations
and adds overhead for safety checks.
% Zimmerman et al 1992: FEM OO

\section{Accessible areas of the code}

Some areas of the code have been written in such a way that a moderately 
experienced Fortran 95 programmer,
with a limited understanding of the code architecture,
can introduce new features without breaking the whole system.
This is achieved through modularity, encapsulation and templates.
The modifications that are currently accessible are:
\begin{itemize}
  \item boundary conditions, see \texttt{bc\_gen.f90}
  \item material rheology, see \texttt{mat\_gen.f90}
%  \item friction laws
  \item source time functions, see \texttt{stf\_gen.f90}
  \item spatial distributions, see \texttt{distribution\_general.f90}
%   \item output fields
\end{itemize}
The source files listed above contain step-by-step instructions,
just follow the comments starting by \texttt{!!}.

