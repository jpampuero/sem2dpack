\chapter{Introduction}

\section{Overview}

The SEM2DPACK package is a set of software tools for the simulation
and analysis of 2D wave propagation and dynamic fracture,
with emphasis on computational seismology and earthquake dynamics.
The core of the package is SEM2D, a solver 
for the 2D elastic wave equations and dynamic earthquake rupture 
based on the Spectral Element Method (SEM) with explicit time stepping.
\charef{phys} of this User's Guide summarizes the range of problems
that can be solved with SEM2DPACK.
\secref{semethod} provides some background on the SEM.
The essential properties of the method are its high order accuracy, 
affordable at competitive computational cost, 
its geometrical flexibility to treat realistic, complicated crustal structures,
and its natural treatment of mixed boundary conditions 
such as fault friction.

SEM2DPACK provides tools for each step
of the general flow of a simulation project:
\begin{enumerate}
  \item {\bf Mesh generation}: partition the domain into (deformed) quadrilateral elements.
Whereas no general mesh generation code is included,
SEM2DPACK contains basic meshing utilities for structured and semi-structured grids
and can import unstructured quadrilateral meshes generated externally.
These features are described in \charef{meshing}.
  \item {\bf Mesh quality verification}: check the accuracy, stability and computational cost, applying tools described in \secref{check}.
Return to previous step if needed. 
  \item {\bf Numerical simulation}: run the SEM2D solver.
\charef{sem2d} explains its usage.
  \item {\bf Post-processing}: visualization and analysis of the output.
A number of post-processing and graphic tools are included,
as decribed in \secref{output}.
Outputs are in the form of binary data files, ASCII data files and PostScript figure files.
Scripts are provided for graphic display and analysis on Seismic Unix, Gnuplot and Matlab.
We recommend the usage of the Matlab functions included (see \secref{matlab}).
\end{enumerate}

This is a research code, constantly under development and provided ``as is'', 
and therefore it should {\em not\/} be considered by the user
as a $100\%$ bug-free software package.
We welcome comments, suggestions, feature requests, bug reports (see \secref{support})
and contributions to the code itself (see \secref{contrib}).

\section{History and credits}

The main part of the elastic-isotropic solver was written by
Dimitri Komatitsch as part of his Ph.D. thesis \cite{Kom97} 
under the direction of Prof. Jean-Pierre Vilotte
at the Institut de Physique du Globe de Paris (IPGP).
The elastic-anisotropic solver and several significant
improvements were added by D. Komatitsch
later as part of a research contract with DIA Consultants.
Further functionalities were added by Jean-Paul Ampuero, 
as part of a Ph.D. thesis \cite{Amp02} also directed by Prof. Vilotte at IPGP.
Most of these additional features were motivated by an ECOS-NORD/FONACYT 
research project for the study of the 
seismic response of the valley of Caracas, Venezuela.
That became the version 1.0 of the SEM2DPACK, released in April 2002.

For version 2.0, most of the solver was rewritten 
in preparation for the implementation of higher level functionalities.
%such as multigrid, subcycling, adaptivity and multiscale coupling.
%While the extensive use of object-oriented features of FORTRAN 9x can degrade performance
%this is not critical in 2D simulations. The emphasis has been 
%rather in facilitating code reuse and extension.
Developments for the simulation of earthquake dynamics \cite{Amp02} 
were included in the main branch of SEM2DPACK in October 2003 (version 2.2). 
Spontaneous rupture along multiple non-planar faults can be currently modelled,
with a range of friction laws.

Non-linear, inelastic materials were introduced in March 2008 (version 2.3). 
Damage and visco-plastic rheologies are included especially
for the modeling of earthquake rupture with off-fault dissipation.

The development of SEM2DPACK has been supported
by the SPICE Research and Training Network of the European Commission,
the National Science Foundation (grant EAR-0944288),
the Southern California Earthquake Center
(funded by NSF Cooperative Agreements EAR-0106924, EAR-0529922, 
and USGS Cooperative Agreements 02HQAG0008, 07HQAG0008)
and URS Corporation.

\section{Application examples}

SEM2DPACK has been utilized in a variety of applications in Earth sciences
and has contributed to more than 20 publications:

\begin{sitemize}
  \item dynamic rupture on non-planar faults and seismic wave radiation \shortcite{MadAmpAdd06} 
  \item wave propagation in anisotropic TTI media \shortcite{Dewetal06}
  \item fault reflections from fluid-infiltrated faults \shortcite{Hanetal07}
  \item benchmark for anisotropic wave propagation \shortcite{Pueetal07}
  \item dynamic earthquake rupture with rate-and-state friction \shortcite{KanLapAmp08} 
  \item benchmark for dynamic earthquake rupture simulation \shortcite{Pueetal09} 
  \item non-linear wave propagation in damaged rocks \shortcite{Lyaetal09} 
  \item modeling marine seismic profiles \shortcite{RobWhiChr09}
  \item surface wave propagation in applied geophysics \shortcite{VigCas10,Vigetal11,Vig12,Boa12} 
  \item wave propagation around a prototype nuclear waste storage tunnel \shortcite{SmiSni10} 
  \item earthquake dynamic rupture with off-fault plasticity \shortcite{Haretal11,Gabetal12b,XuBZAmp12a,XuBZAmp12b} 
  \item earthquake rupture in heterogeneous media \shortcite{HuaAmp11,HuaAmp12b} 
  \item benchmark for wave propagation in heterogeneus media \shortcite{ObrBea11}
  \item dynamic rupture model of the 2011 Tohoku earthquake \shortcite{HuaAmp12a} 
  \item dynamic rupture model of the 2012 off-Sumatra earthquake \shortcite{MenAmp12} 
  \item earthquake rupture with velocity-and-state-dependent friction \shortcite{Gabetal12a} 

\end{sitemize}

\section{Download and updates}

SEM2DPACK is hosted by SourceForge at\\
\centerline{\url{http://sourceforge.net/projects/sem2d/}.}\\
All versions of the code can be downloaded from the package repository at\\
\centerline{\url{http://sourceforge.net/projects/sem2d/files/sem2dpack/}.}

Taking full advantage of the convenient features offered by SourceForge 
(subscribe to new release announcements, submit and track bug reports)
requires a SourceForge.net account, which can be created at\\
\centerline{\url{http://sourceforge.net/account/registration/}.}

SEM2DPACK is updated regularly, typically every few months.
To receive email notifications about new releases you must
sign up for the ``Update Notifications" in the project's main page (scroll down a bit):\\
\centerline{\url{http://sourceforge.net/projects/sem2d/}.}

\section{Requirements}

Compiling the solver code requires the \texttt{make} utility
and a Fortran 95 compiler.
The code is being developed with the Intel compiler for Linux.
It works properly with the Intel compiler starting with
version 8.0.046\_pe047.1, so make sure you have a recent version of \texttt{ifort}. 
Other compilers are not being tested on a regular basis,
so please report any related problems.

The solver runs under the Linux operating system. 
In particular, input/output file name conventions are specific to Linux.
Other operating systems have not been tested.

Pre-processing and post-processing tools, including graphic visualization,
are provided for Seismic Unix, Gnuplot, GMT and Matlab.
The included Matlab tools are by far the most complete,
so a Matlab license is highly recommended.
Matlab ``clone'' softwares have not been tested.

\section{Installation}

\begin{enumerate}
\item Uncompress and expand the SEM2DPACK package:
\texttt{tar xvfz sem2dpack.tgz}
\item Go to the source directory: \texttt{cd SEM2DPACK/SRC} 
\item Edit the \texttt{Makefile} according to your FORTRAN 95 compiler,
following the instructions therein.
\item Modify the optimization parameters declared and described in \texttt{SRC/constant.f90}.
\item Compile: \texttt{make}
\item Move to the \texttt{SEM2DPACK/POST} directory, edit the \texttt{Makefile} and compile.
\end{enumerate}
On normal termination you should end up with a set of executable files, among which
\texttt{sem2dsolve}, in \texttt{/home/yourhome/bin/}.

\section{Documentation}
\label{Sec:docs}

Documentation is available through the following resources:
\begin{sitemize}
  \item This User's Manual
  \item The \texttt{EXAMPLES} directory contains several examples, some have a \texttt{README} file
  \item The pre-processing and post-processing tools for Matlab are documented 
through Matlab's help. For instance \texttt{help mesh2d} provides an overview of
the MESH2D utilities, and \texttt{help mesh2d\_wedge} provides detailed documentation
for the wedge meshing function
  \item The \texttt{ToDo} file contains a list of known issues
\end{sitemize}


\section{Support}
\label{Sec:support}

Support for users of SEM2DPACK is available through a {\it tracking system} at \\
\centerline{\url{http://sourceforge.net/tracker/?group_id=182742},}\\
Three separate {\it tracker lists} deal with the following aspects:
\begin{sitemize}
  \item {\it Feature Requests}: requests for implementation of new features
  \item {\it Support Requests}: questions related to the usage of SEM2DPACK
  \item {\it Bugs}: bug reports
\end{sitemize}
Before submitting an issue make sure that: 
\begin{senumerate}
\item you have read the documentation (see \secref{docs}), 
including the Frequently Asked Questions (\charef{faq}).
Suggestions on how to improve the documentation are also welcome.
\item you are running the most recent version of SEM2DPACK. 
Your issue might have been already fixed in a more recent version.
\item you understand the changes listed in SEM2DPACK's \texttt{ChangeLog} file,
especially changes in the format of the input files
\item your problem has not been treated in previous submissions.
You can browse the tracker message titles or search for keywords.
%By default the browser shows only pending issues, 
%to see the complete list (including resolved issues) set ``Status" to ``Any".
\end{senumerate}
A new submission must include the input files needed to reproduce your problem 
(\texttt{Par.inp}, \texttt{*.ftq}, \texttt{*.mesh2d}, etc). 
You will receive email notifications of any update of your submitted item, until it is closed.
If the item is declared ``Pending'' you are expected to reply to the last message
of the developer within two weeks, otherwise the item will be closed.
For more instructions see\\
\centerline{{\url{http://sourceforge.net/support/getsupport.php?group_id=182742}.}}

\section{Contributions}
\label{Sec:contrib}

Contributions to SEM2DPACK by experienced programmers are always welcome and encouraged.
Although the code is stable for typical applications in computational seismology and
earthquake dynamics, there is still a number of missing features.
Their implementation could make SEM2DPACK interesting for a broader audience in mechanical engineering, 
geotechnical engineering, applied geophysics and beyond. 

The solver code is written in FORTRAN 95.
Resources available for programmers include:
\begin{sitemize}
  \item A \texttt{ToDo} file included with SEM2DPACK
contains a wish list that ranges from basic functionalities to complex code re-engineering.
  \item \charef{add} gives some guidelines for programmers.
  \item A ``Developers Forum" to discuss the implementation of new features is available at \\
\centerline{\url{http://sourceforge.net/forum/forum.php?forum_id=635737},}
\end{sitemize}

We acknowledge the following contributions:
\begin{sitemize}
  \item Shiqing Xu (USC): contributed to the damage material module
  \item Yihe Huang (Caltech): Gaussian source time function, visco-elastic material (attenuation)
  \item Trevor Currie (Caltech): rate-and-state friction and earthquake cycle solver 
% Alice : matlab scripts?
\end{sitemize}

\section{License}
\input{../../LicenseNotice}
