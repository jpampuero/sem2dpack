\chapter{Physical background}
\label{Cha:phys}

This chapter summarizes the physical assumptions and notations in SEM2DPACK.
Footnotes provide reference to the input arguments described in \charef{sem2d}.

\section{General assumptions and conventions}

The coordinate sytem is Cartesian (rectangular). 
SEM2DPACK works in the two-dimensional $(x,z)$ plane,
where $x$ is the horizontal coordinate, with positive direction pointing to the right, 
and $z$ is the vertical coordinate, with positive direction pointing upwards.
The coordinates $(x,y,z)$ will be also denoted as $(x_1,x_2,x_3)$.
This notations carry also for subscripts. For instance,
the $i$-th component of displacement is denoted as $u_i$, with $i=1,2,3$ or with $i=x,y,z$.

The reference frame is Eulerian.
Infinitesimal strain is assumed.
The (symmetric) infinitesimal strain tensor $\varepsilon$ is defined as 
\eq
 \varepsilon_{ij} = \frac{1}{2} \left(\frac{\partial u_i}{\partial x_j} +\frac{\partial u_j}{\partial x_i}\right)
\en
%Dynamic fault boundary conditions assume also small displacements....?

Material density is deonted $\rho(x,z)$.
The displacements and stresses relative to an initial equilibrium configuration
are denoted $u_i(x,z,t)$ and $\sigma_{ij}(x,z,t)$, respectively.
External forces (sources) are denoted $f_i(x,z,t)$. 
SEM2DPACK solves the following equations of motion to obtain
the relative displacements $u_i(x,z,t)$:
\eq
  \rho\ \frac{\partial^2 u_i}{\partial t^2} = \frac{\partial\sigma_{ij}}{\partial x_j} + f_i
\en
where summation over repeated indices is assumed.
The initial conditions are $u_i=0$ and $\partial u_i/\partial t = 0$.
Stresses are related to strain, and possibly to other internal variables,
by constitutive equations described in \secref{rheol}.
The governing equations are supplemented by boundary conditions, described in \secref{bcs}. 
SEM2DPACK actually solves the governing equations in variational (weak) form,
as described in any textbook on the finite element method.

Two types of 2D problems are solved\footnote{In the \texttt{\&GENERAL} input block,
plane strain is selected by $\texttt{ndof=2}$ and antiplane shear by $\texttt{ndof=1}$.}: 
\begin{sitemize}
\item \emph{Plane strain}: 
Also known in seismology as P-SV, and in fracture mechanics as inplane mode or mode II.
It is assumed that $u_3=0$ and $\partial / \partial x_3 = 0$.
Hence, $\varepsilon_{13}=\varepsilon_{23}=\varepsilon_{33}=0$
and there are two degrees of freedom per node, $u_x$ and $u_z$.
\item \emph{Antiplane shear}: 
Also known in seismology as SH, and in fracture mechanics as antiplane mode or mode III.
It is assumed that $u_1=u_2=0$ and $\partial / \partial x_3 = 0$. 
Hence, only $\varepsilon_{13}$ and $\varepsilon_{23}$ are non-zero
and there is one degree of freedom per node, $u_y$.
\end{sitemize}

\section{Material rheologies}
\label{Sec:rheol}

We describe here the constitutive equations implemented in SEM2DPACK,
relating stress ($\sigma_{ij}$), strain ($\varepsilon_{ij}$) and internal variables.

\subsection{Linear elasticity}

\subsubsection{Linear isotropic elasticity}

Stress and strain are linearly related by Hooke's law, 
$\sigma_{ij} = c_{ijkl} \varepsilon_{ij}$,
where $c_{ijkl}$ is the tensor of elastic moduli.
In particular, for isotropic elasticity:
\eq
  \sigma_{ij} = \lambda \varepsilon_{kk} \delta_{ij} + 2\mu \varepsilon_{ij}
\en
where $\lambda$ and $\mu$ are Lam\'e's first and second parameters, respectively.
In 2D plane strain the only relevant stress components are $\sigma_{11}$, $\sigma_{22}$ and $\sigma_{12}$.
The intermediate stress $\sigma_{33}$, although not null, does not enter in the equations of motion.
The S and P wave speeds are $c_S = \sqrt{\mu/\rho}$ and $c_P = \sqrt{(\lambda+2\mu)/\rho}$, respectively.
In 2D antiplane shear only the stress components $\sigma_{13}$ and $\sigma_{23}$ are relevant,
and only S waves are generated.

\subsubsection{Linear anisotropic elasticity}

Transverse anisotropy with vertical symmetry axis (VTI)
is implemented for 2D P-SV \shortcite{KoBaTr00b}. % and SH modes.
The stress-strain constitutive relation for P-SV in Voigt notation is:
\eq
  \left( \begin{array}{c}
    \sigma_{xx} \\ \sigma_{zz} \\ \sigma_{xz}
  \end{array} \right)
  = 
  \left( \begin{array}{ccc}
    c_{11} & c_{13} & 0 \\
    c_{13} & c_{33} & 0 \\
    0      & 0      & c_{55}
  \end{array} \right) \ 
  \left( \begin{array}{c}
    \varepsilon_{xx} \\ \varepsilon_{zz} \\ 2\varepsilon_{xz}
  \end{array} \right)
\en
where the $c_{ij}$ are elastic moduli. 
For SH:
\eq
  \left( \begin{array}{c}
    \sigma_{yz} \\ \sigma_{xy} 
  \end{array} \right)
  = 
  \left( \begin{array}{cc}
    c_{55} & 0 \\
    0 & c_{66} 
  \end{array} \right) \ 
  \left( \begin{array}{c}
    2 \varepsilon_{yz} \\ 2\varepsilon_{xy}
  \end{array} \right)
\en


\subsection{Linear visco-elasticity}

\subsubsection{Generalized Maxwell material}
\emph{Not implemented yet.}

\subsubsection{Kelvin-Voigt material}

Kelvin-Voigt damping can be combined with any of the other constitutive equations
by replacing the elastic strain $\varepsilon$ by $\varepsilon^* = \varepsilon + \eta\,\partial\varepsilon/\partial t$,
where $\eta$ is a viscosity timescale.

The resulting quality factor $Q$ is frequency-dependent, $Q^{-1}(f) = 2\pi\eta f$.
This rheology is not approriate to model crustal attenuation with constant $Q$, 
unless the source has a narrow frequency band
and $\eta$ is selected to achieve a given $Q$ value at the dominant frequency of the source.

The main application of Kelvin-Voigt viscosity is the artificial damping of high-frequency
numerical artifacts generated by dynamic faults.
Dynamic source simulations using methods that discretize the bulk, such as
finite difference, finite element and spectral
element methods, are prone to high frequency numerical noise 
when the size of the process zone is not well resolved.
Efficient damping is typically achieved
by a thin layer of Kelvin-Voigt elements surrounding the fault,
with thickness of 1 or 2 elements on each side of the fault
and $\eta/\Delta t_{fault} = 0.1$ to $0.3$, where $\Delta t_{fault}$ 
is the critical time step size based on the size of the spectral elements along the fault
(not necessarily equal to the critical time step over the whole mesh).
The value of $\Delta t_{fault}$ can be obtained with 
the Matlab function \texttt{PRE/critical\_timestep.m}.

%Methods to control this problem were presented in the author's
%Ph.D. dissertation \cite{Amp02}\footnote{Available in French at 
%\url{http://web.gps.caltech.edu/~ampuero/publications.html}}
%and in Gaetano Festa's Ph.D. dissertation\footnote{\url{http://people.na.infn.it/~festa/}},
%%In particular the method of Appendix 4.A, a high/low frequency 
%%decomposition with hybrid and consistent asymptotic/SEM treatment of friction,
%%has been tested by Gaetano Festa and J.-P. Vilotte
%and will be implemented in a forthcoming version of SEM2DPACK.

\subsection{Coulomb plasticity and visco-plasticity}

\subsubsection{Perfect plasticity}

Perfect plasticity with a Coulomb yield function is implemented
for 2D plane strain, as in \citeN{And05}.

The total strain is the sum of an elastic and a plastic contribution,
$\varepsilon = \varepsilon^e + \varepsilon^p$.
The plastic strain is assumed to be purely deviatoric ($\varepsilon^p_{kk} = 0$).
Plastic yield occurs when
the maximum shear stress over all orientations,
\eq
  \tau_{max} = \sqrt{ \sigma_{xz}^2 + (\sigma_{xx}-\sigma_{zz})^2/4},
\en
reaches the yield strength,
\eq
  Y = c\,\cos(\phi) - (\sigma_{xx}+\sigma_{zz}) \, \sin(\phi)/2,
\en
where $c$ is the cohesion and $\phi$ is the internal friction angle.

\subsubsection{Visco-plasticity}

In classical Duvaut-Lions visco-plasticity the (visco-)plastic strain rate is proportional
to the excess of stress over the yield strength:
\eq
  \dot{\varepsilon}^p_{kl} = \frac{1}{2\mu\ T_v} \langle\tau_{max} - Y\rangle\ \frac{\tau_{ij}}{\tau_{max}}
\en
where $T_v$ is the visco-plastic relaxation time,
$\langle x\rangle \doteq (x+|x|)/2$ is the ramp function
and $\tau_{ij} = \sigma_{ij} - \frac{1}{3}\sigma_{kk}\ \delta_{ij}$ 
is the deviatoric part of the stress tensor.

Visco-plasticity is often employed as a regularization of plasticity
to avoid or delay the occurrence of strain localization features,
such as shear bands, that are mesh-dependent.
For that particular application, $T_v$ is typically set to 
the average P wave traveltime across a few grid points, 
i.e. a few times the average spacing between GLL nodes divided by the P wave speed.

%----------------------------------------------------------
\subsection{Attenuation}
Following \shortciteN{Mocetal04}, we incorporate attenuation 
by adding viscoelastic terms in the stress-strain relations.
For the sake of simplicity, we present here the formulation in terms of scalar stress and strain:
\eqa
\sigma(t) &=& M_u \varepsilon(t)-\sum_{l=1}^n M_u Y_l \zeta_l(t) \\
\dot{\zeta}_l(t)+\omega_l \zeta_l(t) &=& \omega_l \varepsilon_l(t)
\label{Eq:Qf}
\ena
where $M_u$ is the unrelaxed modulus, $n$ is the number of anelastic mechanisms
and $Y_l$, $\zeta_l(t)$ and $\omega_l$ are the anelastic coefficient, 
anelastic state variable and relaxation frequency of the $l^{th}$ viscoelastic mechanism, respectively. 

The material parameters $Y_l$ and $\omega_l$ are determined to achieve
a prescribed constant $Q$ value within a prescribed frequency band $[\omega_1,\omega_2]$.
The quality factor of this viscoleastic material, as a function of angular frequency $\omega$, is
implicitely given by the solution of the following equation:
\eq
Q^{-1}(\omega)- \sum_{l=1}^n \frac{\omega_l \omega +\omega_l^2 Q^(-1)(\omega)}{\omega_l^2+\omega^2 } Y_l = 0
\en
The relaxation frequencies $\omega_l$ are distributed uniformly in logarithmic scale
within $[\omega_1,\omega_2]$.
The anelastic coefficients $Y_l$ are determined by satisfying Equation~\ref{Eq:Qf} with prescribed $Q$ 
in a least squares sense at $2n-1$ frequencies 
distributed uniformly in logarithmic scale within $[\omega_1,\omega_2]$.
The unrelaxed moduli $M_u$ are derived from the anelastic coefficients, 
following equations 166 and 167 of \shortciteN{Mocetal04}),
by constrainging the phase velocity at the reference frequency $\omega_r=\sqrt{\omega_1 \omega_2}$
to equal the prescribed elastic wave velocity.

For $n=3$, an almost constant $Q$ is achieved, with less than $5\%$ error, 
over a frequency band with maximum to minimum frequency ratio $\sim 100$. 
A benchmark of our attenuation implementation is presented in section 5.1 of \shortciteN{Huaetal14b}.

%----------------------------------------------------------
\subsection{Continuum damage}

The continuum damage formulation by \shortciteN{Lyaetal97a}, 
including damage-related plasticity as introduced by \shortciteN{Hametal04},
is implemented with modifications for 2D plane strain.
%and XXX $\beta$ as introduced by \shortciteN{Hametal04}.

The first and second invariants of the 2D elastic strain tensor are defined as
$I_1 = \varepsilon^e_{kk}$ and $I_2 = \varepsilon^e_{ij}\varepsilon^e_{ij}$, respectively. 
A strain invariant ratio is defined as $\xi = I_1/\sqrt{I_2}$.
The following non-linear stress-strain relation is assumed
\shortcite[eq. 12]{Lyaetal97a}:
\eq
  \sigma_{ij} = ( \lambda - \gamma/\xi )\ I_1 \delta_{ij} 
              + ( 2\mu - \gamma \xi )\ \varepsilon^e_{ij}
\en
where $\gamma$ is an additional elastic modulus.
The elastic moduli depend
on a scalar damage variable, $0\le\alpha\le1$, through \shortcite[eq. 19]{Lyaetal97a}: 
%(\shortciteN[eq. 19]{Lyaetal97a} and \shortciteN[eq. 3]{Hametal04}): % for beta>0
\eqa
  \lambda &=& \lambda_0 \\
  \mu &=& \mu_0 + \gamma_r \xi_0\ \alpha \\
  %\gamma &=& \gamma_r\ \frac{\alpha^{1+\beta}}{1+\beta}
  \gamma &=& \gamma_r\ \alpha
\ena
where $\lambda_0$ and $\mu_0$ are Lam\'e's parameters for the intact material ($\alpha=0$).
The parameter $\xi_0$ is the threshold value of the strain invariant ratio $\xi$
at the onset of damage.
It is related to the internal friction angle $\phi$ in
a cohensionless Mohr-Coulomb yield criterion by 
the 2D plane strain version of \shortciteN[eq. 37]{Lyaetal97a}:
\eq
  \xi_0 = \frac{-\sqrt{2}}{\sqrt{1+ (\lambda_0/\mu_0 +1)^2\ \sin^2\phi}} % in 2D
  %\xi_0 = \frac{-\sqrt{3}}{\sqrt{ 2q^2 (\lambda_0/\mu_0 +2/3)^2 +1}}    % in 3D
\en
%where
%\eq
%  q = \frac{\sin(\phi)}{1-\sin(\phi)/3}
%\en
%The parameter $0<\beta<1$ was introduced by \shortcite{Hametal04}.
The scaling factor $\gamma_r$ is chosen such that convexity is lost
at $\alpha=1$ when $\xi=\xi_0$.
It is derived from the 2D plane strain version of \shortciteN[eq. 15]{Lyaetal97a}:
\eq
 \gamma_r = p+\sqrt{p^2 +2\mu_0 q}
\en
where
\eqa
 q & =& (2\mu_0+2\lambda_0)/(2-\xi_0^2) \\  % in 2D
 %q & =& (2\mu_0+3\lambda_0)/(3-\xi_0^2) \\ % in 3D
  p &=& \xi_0 (q + \lambda_0)/2 
\ena
The evolution equation for the damage variable is
\shortcite[eq. 20]{Lyaetal97a}
%(\shortciteN[eq. 20]{Lyaetal97a} and \shortciteN[eq. 7]{Hametal04})
\eq
  \dot{\alpha} = 
   \left\{
   \begin{array}{ll}
     %C_d I_2 (\alpha^\beta \xi - \xi_0)& \mbox{if $\xi>\xi_0/\alpha^\beta$} \\
     C_d I_2 (\xi - \xi_0)& \mbox{if $\xi>\xi_0$} \\
     0  & \mbox{otherwise}
   \end{array}
  \right.
\en
No healing is assumed below $\xi_0$.
The evolution of the plastic strain $\varepsilon^p_{ij}$ is 
driven by the damage variable $\alpha$ 
\shortcite[eq. 9]{Hametal04}:
\eq
  \dot{\varepsilon}^p_{ij} = 
   \left\{
   \begin{array}{ll}
     \tau_{ij} C_v \dot{\alpha} & \mbox{if $\dot{\alpha}\ge 0$} \\
     0  & \mbox{otherwise}
   \end{array}
  \right.
\en
where $\tau_{ij} = \sigma_{ij} - \frac{1}{3}\sigma_{kk}\ \delta_{ij}$ 
is the deviatoric part of the stress tensor.
The parameter $C_v$ is of order $1/\mu_0$
and\footnote{In \texttt{\&MAT\_DMG}, the input argument \texttt{R} is defined as $R=\mu_0 C_v$.}
is related to the seismic coupling coefficient $0<\chi<1$ by 
\cite{BZLya06}
\eq
  C_v = \frac{1-\chi}{\chi}\ \frac{1}{\mu_0}
\en
%and the factor $R = \mu_0 C_v$ is related to the seismic coupling coefficient $0<\chi<1$ by $\chi=1/(1+R)$
%$\chi=1/(1+\mu_0 C_v)$

%----------------------------------------------------------
%----------------------------------------------------------
\section{Boundary conditions}
\label{Sec:bcs}

\subsection{Absorbing boundaries}

Two approximate absorbing boundary conditions (ABC)
are implemented to model the outwards propagation of waves 
at the boundaries of the computational domain.
Both conditions are of paraxial type. Their performance is appropriate at normal incidence
but degrades at grazing incidence. 

\subsubsection{Clayton-Engquist ABC}

In the local coordinate frame $(t,n)$ related to the tangential ($t$)
and outgoing normal ($n$) directions to the boundary, the
first-order accurate ABC proposed by \citeN{ClaEng77} reads:
\eqa
  \dot{u}_t &=& - c_S \frac{\partial u_t}{\partial x_n} \\
  \dot{u}_n &=& - c_P \frac{\partial u_n}{\partial x_n}
\ena
The implementation is based on an equivalent formulation 
as a mixed boundary condition
that relates tractions $T$ to displacements $u$:
\eqa
  T_t &=& - \rho c_S\ \dot{u}_t \\
  T_n &=& - \rho c_P\ \dot{u}_n
\ena
The formulation above is for P-SV mode. 
In SH mode the ABC is $T_y= - \rho c_S\ \dot{u}_y$.

\subsubsection{Stacey ABC}

The second-order accurate ABC introduced by \citeN{Sta88} 
under the name ``P3'' is:
\eqa
  \dot{u}_t &=& - c_S \frac{\partial u_t}{\partial x_n}
                - (c_P-c_S) \frac{\partial u_n}{\partial x_t}\\
  \dot{u}_n &=& - c_P \frac{\partial u_n}{\partial x_n}
                - (c_P-c_S) \frac{\partial u_t}{\partial x_t}
\ena
Its formulation as a mixed boundary condition is:
\eqa
  T_t &=& - \rho c_S\ \dot{u}_t + \rho c_S (2 c_S-c_P)\ \frac{\partial u_n}{\partial x_t}\\
  T_n &=& - \rho c_P\ \dot{u}_n - \rho c_S (2 c_S-c_P)\ \frac{\partial u_t}{\partial x_t}
\ena
This ABC is only implemented in P-SV mode.

\section{Fault interface conditions}

\subsection{Linear slip law}
Represents a compliant fault zone with elastic contact.
See \shortciteN{Hanetal07}.
[...]

\subsection{Normal stress response}

\subsubsection{Unilateral contact}

No interpenetration during contact, free stress during opening.
[...]

\subsubsection{Modified Prakash-Clifton regularization}

Regularization of the normal stress response, as required for
bimaterial rupture problems, is implemented following \citeN{RubAmp07}.
The frictional strength is proportional to a modified normal stress $\sigma^*$,
related to the real fault normal stress, $\sigma$,
by either of the following evolution laws:
\begin{sitemize}
 \item Version with a regularization \emph{time} scale:
\eq
  \dot{\sigma^*} = \frac{1}{T_\sigma}\ (\sigma - \sigma^*)
\en
where $T_\sigma$ is a constitutive parameter\footnote{In \texttt{\&BC\_DYNFLT\_NOR},
this law is set by \texttt, and the relevant parameter is \texttt{T}.}.
 \item Version with a regularization \emph{slip} scale:
\eq
  \dot{\sigma^*} = \frac{|V| + V^*}{L_\sigma}\ (\sigma - \sigma^*)
\en
where $V$ is slip rate, 
and $V^*$ and $L_\sigma$ are constitutive parameters\footnote{In \texttt{\&BC\_DYNFLT\_NOR},
this law is set by \texttt, and the two relevant parameters are \texttt{V} and \texttt{L}.}.
\end{sitemize}

\subsection{Friction}

\subsubsection{Slip-weakening friction}
Slip occurs when the fault shear stress reaches the shear strength $\tau = \mu\sigma$ 
(or $\tau=\mu\sigma^*$ if the Prakash-Clifton law is assumed). [...]
The friction coefficient $\mu$ is a function of the cumulated slip $D$,
given by one of the following laws:
\begin{sitemize}
  \item Linear slip-weakening law:
\eq
  \mu = \max\left[ \mu_d , \mu_s - \frac{\mu_s-\mu_d}{D_c}\ D \right]
\en
  \item Exponential slip-weakening law:
\eq
  \mu = \mu_s - (\mu_s-\mu_d)  \exp(-D/D_c)
\en
\end{sitemize}

\subsubsection{``Fast'' rate-and-state-dependent friction}

Friction with fast (power law) velocity weakening at fast slip speed
is a first order proxy for physical weakening processes that operate
on natural fault zones at coseismic slip velocities.
A rate-and-state dependent friction law with fast velocity-weakening is implemented in SEM2DPACK,
similar to that adopted e.g. by \citeN{AmpBZ08}.
The friction coefficient depends on slip velocity ($V$) and a state variable ($\theta$):
\eq
  \mu_f = \mu_s +a\ \frac{V}{V+V_c} - b\ \frac{\theta}{\theta+D_c}.
  \label{Eq:muf}
\en
The state variable has units of slip and is governed by the evolution equation
\eq
  \dot{\theta} = V-\theta V_c/D_c.
\label{Eq:th}
\en
The friction law is defined by the following constitutive parameters:
$\mu_s$ is the static friction coefficient,
$a$ and $b$ are positive coefficients of a direct effect and an evolution effect, respectively,
$V_c$ is a characteristic velocity scale\footnote{\texttt{vstar} in \texttt{\&BC\_DYNFLT\_RSF}},
and $D_c$ is a characteristic slip scale.

The steady-state ($\dot{\theta}=0$) friction coefficient 
\eq
  \mu_f = \mu_s +(a-b)\ \frac{V}{V+V_c}
\en
weakens asymptotically as $1/V$ when $V\gg V_c$,
if $a<b$, approaching its dynamic value ($\mu_d = \mu_s + a-b$)
over a relaxation timescale $D_c/V_c$.
The value of the relaxation time $D_c/V_c$ tunes the weakening mechanism between two limit cases:
slip-weakening and velocity-weakening.
If $D_c/V_c$ is much longer than the typical time scale of fluctuation of the
state variable ($\approx \theta/\dot{\theta}$),
\Eqref{th} becomes $\dot{\theta} \approx V $,
implying that $\theta$ is proportional to slip
and that the evolution term of the friction coefficient
is effectively slip-weakening, with characteristic slip-weakening distance $D_c$.
Conversely, if $D_c/V_c$ is short the relaxation to steady state is fast,
$\theta/D_c \approx V/V_c$
and the friction is effectively velocity-weakening,
with characteristic velocity scale $V_c$.

\subsubsection{Logarithmic rate-and-state friction}
\emph{Not implemented yet.}

Dieterich and Ruina classical rate-and-state laws, with aging or slip state evolution law.
