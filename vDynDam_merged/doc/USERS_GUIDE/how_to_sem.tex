\chapter{The solver SEM2D}
\label{Cha:sem2d}

\section{About the method}
\label{Sec:semethod}

Given a crustal model meshed with quadrilateral elements
and a set of material properties, sources, receivers and boundary conditions,
SEM2D solves the elastic wave equation applying 
the Spectral Element Method (SEM) for the space discretization
and a second-order explicit scheme for the time discretization.
The range of physical problems solved by SEM2D
(material constitutive equations and boundary conditions)
is described in more detail in \charef{phys}.
The SEM, introduced by \citeN{Pat84} in Computational Fluid Dynamics,
can be seen as a domain decomposition version of the Spectral Method
or as a high order version of the Finite Element Method.
It inherits from its parent methods the accuracy (spectral 
convergence), the geometrical flexibility 
and the natural implementation of mixed boundary conditions.

Introductory texts to the SEM can be found 
at \url{www.math.lsa.umich.edu/~karni/m501/boyd.pdf} 
(chapter draft, by J.P. Boyd), at
\url{www.mate.tue.nl/people/vosse/docs/vosse96b.pdf}
(a tutorial exposition of the SEM and its connection
to other methods, by F.N. van de Vosse and P.D. Minev)
and at \url{www.siam.org/siamnews/01-04/spectral.pdf} (a perspective paper).
Details about the elastodynamic algorithm and study of some of its properties
are presented by \shortciteN{Kom97}, \shortciteN{KoVi98}, \shortciteN{KoVi99},
\citeN{KomTro99} and \shortciteN{VaCaSaKoVi99}.

The implementation of fault dynamics is similar to that in FEM
with the ``traction at split nodes" method explained by \citeN{And99}.
More details can be found in the author's 
Ph.D. dissertation \cite{Amp02}\footnote{
\url{web.gps.caltech.edu/~ampuero/publications.html}},
in Gaetano Festa's Ph.D. dissertation\footnote{\url{people.na.infn.it/~festa/}}
and in \shortciteN{KanLapAmp08}.

A more accesible tutorial code, SBIEMLAB written in Matlab, 
can be downloaded from the author's website, at
\url{web.gps.caltech.edu/~ampuero/software.html}.

\section{Basic usage flow}

In general, a simulation requires the following steps:
\begin{senumerate}
  \item Prepare the input file \texttt{Par.inp} (\secref{input} and \secref{inblo}).
  \item Run the solver in ``check mode'' (\texttt{iexec=0} in the \texttt{GENERAL} input block of \texttt{Par.inp}):
        \texttt{sem2dsolve > info \&}.
  \item Verify the resolution, stability, estimated CPU cost and memory cost (\secref{check}).
  \item If needed go back to step 1 and modify \texttt{Par.inp} (\secref{check}), else proceed to next step.
  \item Run the solver in ``production mode'' (\texttt{iexec=1}): \texttt{sem2dsolve}.
  \item Plot and manipulate the solver results (\secref{output}).
\end{senumerate}
Full details are given in the following sections.

\section{General format of the input file}
\label{Sec:input}

The input file must be called \texttt{Par.inp}.
Its typical structure is illustrated by two examples
in \figref{ParInp1} and \figref{ParInp2}.
Most of the file is made of standard FORTRAN 90 NAMELIST input blocks.
Each block gives input for a specific aspect of the simulation:
material properties, sources, receivers, boundary conditions, etc.

The general syntax of a NAMELIST block can be found in any FORTRAN 90 textbook.
In summary, a block named \texttt{STUFF}, with possible input arguments 
\texttt{a}, \texttt{b} and \texttt{c}, must be given as
\begin{verbatim}
 &STUFF a=..., b=..., c=... /
\end{verbatim}
where \texttt{...} are user input values. 
Line breaks and comments preceded by \texttt{!} are allowed within an input block.

The complete Reference Guide of the input blocks is presented in \secref{inblo}.
For each block the documentation includes its name, 
possibly the name of a group of blocks to which it belongs,
its purpose, its syntax, the list of its arguments with their description, and some important notes.
In the syntax description, a vertical bar \texttt{|} between two arguments means ``one or the other''.
In the argument list, each item is followed by two informations within brackets \texttt{[]}.
The first bracketed information is the type of the argument: 
double precision (\texttt{dble}), integer (\texttt{int}),
logical (\texttt{log}), single character (\texttt{char}), 
fixed length word (e.g. \texttt{char*6} is a 6 characters word),
arbitrary length word (\texttt{name}) or vectors (e.g. \texttt{int(2)} is a two element integer vector).
The second bracketed information is the default value of the argument.
Some arguments are optional, or when absent they are automatically assigned the default values.

Some arguments have a second version with a suffix \texttt{H}
that allows to set values that are spatially non uniform.
The \texttt{H}-version of the argument must be set to the name of any of the input blocks
of the \texttt{DISTRIBUTIONS} group.
The appropriate \texttt{\&DIST\_xxxx} block must follow immediately.
For example, to set the argument \texttt{eta} to a Gaussian distribution:
\begin{verbatim}
&MAT_KV etaH='GAUSSIAN' /
&DIST_GAUSSIAN length=1d6,100d0, ampli=0.1d0 /
\end{verbatim}
Arguments that accept an \texttt{H}-version are indicated in \secref{inblo}.
When more than one \texttt{H}-version argument is present,
the \texttt{\&DIST\_xxxx} blocks must appear in the same order as in
the argument list of \secref{inblo}.

In the next section, Input Block Reference Guide,
you should get acquainted with the syntax of the blocks you are most likely to use.
The mandatory or more important input blocks are:

\begin{sitemize}
\item \texttt{\&GENERAL}
\item \texttt{\&MESH\_DEF}, followed by a \texttt{\&MESH\_Method} block
\item \texttt{\&MATERIAL}, followed by a \texttt{\&MAT\_Material} block
\item \texttt{\&BC\_DEF}, one for each boundary condition, each followed by a \texttt{\&BC\_Kind} block
\item \texttt{\&TIME}
\item \texttt{\&SRC\_DEF}, followed by \texttt{\&STF\_SourceTimeFunction} and \texttt{\&SRC\_Mechanism} blocks
\item \texttt{\&REC\_LINE}
\end{sitemize}

\begin{figure}[p]
\ImgC{Par.inp.ex1.ps}{0.95}
\caption{\label{Fig:ParInp1} Input file \texttt{Par.inp} for an
elementary example in \texttt{EXAMPLES/TestSH/}
: a boxed region with a structured mesh.}
\end{figure}

\begin{figure}[p]
\ImgC{Par.inp.ex2.ps}{0.95}
\caption{\label{Fig:ParInp2} Input file \texttt{Par.inp} for a
more realistic example: a sedimentary basin with an unstructured mesh generated
by \texttt{EMC2}. Available in \texttt{EXAMPLES/UsingEMC2/}.}
\end{figure}
 
 \newpage
\section{Input Blocks Reference Guide}
\label{Sec:inblo}
%% vertical spacing
\renewcommand\arraystretch{1.5}% (=1.0 is for standard spacing)

\newenvironment{my_enumerate}{
\begin{enumerate}
  \setlength{\itemsep}{1pt}
  \setlength{\parskip}{0pt}
  \setlength{\parsep}{0pt}}{\end{enumerate}
}

\subsection{Sources}

\subsubsection{SRC\_DEF}
{\it Purpose:} Define the sources.

{\it Syntax:} \texttt{\&SRC\_DEF   TimeFunction,mechanism,coord /}\\
followed by blocks of the groups SRC\_TIMEFUNCTION and SRC\_MECHANISM 

{\it Arguments:}

\begin{tabular}[t]{lllp{0.5\linewidth}}
\texttt{TimeFunction} &  name& none& The name of the source time function: 'RICKER', 'TAB' or 'STF\_USER' \\
\texttt{mechanism} & name & none & The name of the source mechanism: 'FORCE', 'EXPLOSION', 'DOUBLE\_COUPLE', 'MOMENT' or 'WAVE' \\
\texttt{coord}  & dble & huge & Location of the source (m). \\
\texttt{file}  & string & 'none' & 
% \begin{minipage}[t]{\linewidth}
Station coordinates and delay times can
be read from an ASCII file, with 3 columns per line:
\begin{my_enumerate}
  \item X position (in m),
  \item Z position (in m) and
  \item delay (in seconds) 
\end{my_enumerate}
that's it.
%\end{minipage}
\end{tabular}

{\it Notes:}
\begin{enumerate}
  \item bla bla 
  \item bla bla
\end{enumerate}


\input{selfdoc.tex}
 \newpage

\section{Verifying the settings and running a simulation}
\label{Sec:check}
 
Once the code has been successfully compiled, the simulation
can be started by typing \texttt{sem2dsolve} from your working directory, 
which contains the file \texttt{Par.inp}.
The computations can be run in background
and the screen output saved in a file (e.g. \texttt{info}) by typing
\texttt{sem2dsolve > info \&}.

A typical screen output of SEM2D, corresponding
to the first example, is shown on the following pages.
The parameters of the simulation 
and some verification information are reported there in a self-explanatory form. 
You are advised to
do a first run with \texttt{iexec=0} in the \texttt{GENERAL} input block
and check all these informations prior to the real simulation.
You should always verify the following:

\begin{itemize}

\item {\bf Stability:}
the CFL stability number should be smaller than $0.55 \sim 0.60$ for second order time schemes,
and much smaller for highly deformed meshes (see Section \ref{faq_sem2d} on
``Instabilities in very distorted elements'').
This number is defined at each computational node as $$\mbox{CFL} = c_P\ \Delta t /\Delta x$$
where $\Delta t$ is the timestep, 
$c_P$ the P-wave velocity and $\Delta x$ the local grid spacing. 
Note that $\Delta x$ is usually much smaller than the element size $h$
($\approx$ \texttt{Ngll}$^2$ times smaller)
because SEM internally subdivides each element onto a non-regular
grid of \texttt{Ngll$\times$Ngll} nodes clustered near the element edges
(Gauss-Lobatto-Legendre nodes).
If the computation is unstable, the maximum displacement,
printed every \texttt{ItInfo} time steps, increases exponentially with time.
Stability can be controlled by decreasing \texttt{Dt} 
or \texttt{Courant} in \texttt{Par.inp}.

\item {\bf Resolution:}
the number of nodes per shortest wavelength $\lambda_{min}$ should be larger than $4.5 \sim 5$.
The minimum wavelength is defined as
$$\lambda_{min}=\min(c_S)/f_{max}$$ 
where $c_S$ is the S-wave velocity and 
$f_{max}$ the highest frequency you would like to resolve, 
e.g. the maximum frequency at which the source spectrum has significant power
(for a Ricker wavelet $f_{max} = 2.5\times f_0$).
For an element of size $h$ 
and polynomial order $p=\texttt{Ngll}-1$, the number of nodes per wavelength $G$
is $$G = \frac{p\,\lambda_{min}}{h}.$$
Typical symptoms of poor resolution are ringing and dispersion of the higher frequencies.
However, in heterogeneous media
these spurious effects might be hard to distinguish from a physically complex wavefield,
so mesh resolution must be checked beforehand. 
If resolution is too low the mesh might be refined by 
increasing \texttt{Ngll} in \texttt{Par.inp} ($p$-refinement) 
or by generating a denser mesh ($h$-refinement).
If you were using EMC2 as a mesh generator,
the script \texttt{PRE/href.csh} can be useful for $h$-refinement.

\item {\bf Cost:}
the total CPU time an memory 
required for the simulation are as much as you can afford.
Estimates of total CPU time are printed at the end of check mode.
Details about memory usage can be found in \texttt{MemoryInfo\_sem2d.txt}.

\end{itemize}

\begin{figure}
\ImgC{Stab_Reso_ex.eps}{1}
\caption{\label{Fig:stabres} Checking the quality of a mesh
with \texttt{PRE/ViewMeshQuality.m} for the example in \texttt{EXAMPLES/UsingEMC2/}.
The balance of the stability and resolution properties of the mesh can be analyzed:
logarithmic stability index (top) and logarithmic resolution index (bottom).
Histograms of these indices (in number of elements) are shown on the right.}
\end{figure}

The quality of the mesh can be inspected with
the Matlab script \texttt{PRE/ViewMeshQuality.m} which produces
plots like Figure \ref{Fig:stabres}.
The proper balance of the mesh with respect to the following
two criteria can be analyzed:

\begin{itemize}

\item {\bf Stability criterion}, related to the largest stable timestep. 
The stability of each element is quantified by
$$ S = \min(\Delta x/c_P). $$
We also define a stability index as 
$$ SI = \log[ S / \text{median}(S) ].$$
where the median value is taken over the whole mesh.
Red elements (small SI) are relatively unstable
and require small timesteps $\Delta t$.
Because $\Delta t$ is constant over the whole mesh
and the computational cost is inversely proportional to $\Delta t$,
these red elements penalize the computational efficiency.
The mesh should be redesigned to increase their size, as much as possible,
while keeping them small enough to resolve the shortest wavelength (see next).
%Conversely, elements with very high SI (blue) could in principle be smaller.

\item {\bf Resolution criterion},
related to the number of nodes per shortest wavelength. 
The resolution of each element is quantified by
$$ R = \min(c_S / h). $$
We also define a resolution index as 
$$ RI = \log[ R / \text{median}(R) ].$$
where the median value is taken over the whole mesh.
Red elements (small RI) have relatively poor resolution,
in their vicinity the maximum frequency resolvable by the mesh is limited.
The mesh should be redesigned to decrease their size, as much as possible.
Conversely, elements with very high RI (blue) are smaller than required
and might increase the computational cost.

\end{itemize}

To minimize the CPU and memory cost of a simulation
an ideal mesh design should minimize the spread of the two indices above,
by aiming at a ratio of element size to wave velocity, $h/c$, 
as uniform as possible across the whole mesh.
%Resolution and stability are sometimes competing constraints in mesh design,
%trade-offs must be resolved through experience.
%To improve the quality of your mesh 
%try to use larger elements in the regions where very low stability was observed
%but keep them large enough to resolve the shortest wavelength.
However, in some cases a poorly balanced mesh is inevitable: in the example 
of Figure \ref{Fig:stabres} the worst elements are near the edges of the
sedimentary basin, at a sharp velocity contrast.
Small element sizes on the rock side are inherited from the sediment mesh.\footnote{In 
future releases of SEM2DPACK
this penalty on computational efficiency will be reduced by 
non-conformal meshing with mortar elements,
by timestep subcycling or by implicit/explicit timestep partitioning.}

Similar information is plotted by
\texttt{gv Stability\_sem2d.ps} and \texttt{gv Resolution\_sem2d.ps}.
The indices in these files are however not logarithmic and are
not normalized by the median.\\\

% include this file in your .tex :
%   % include this file in your .tex :
%   % include this file in your .tex :
%   \input{info.ex1.ps.tex}
\ImgC{info.ex1_1.ps}{1}
\ImgC{info.ex1_2.ps}{1}

\ImgC{info.ex1_1.ps}{1}
\ImgC{info.ex1_2.ps}{1}

\ImgC{info.ex1_1.ps}{1}
\ImgC{info.ex1_2.ps}{1}

 
\section{Outputs, their visualization and manipulation}
\label{Sec:output}

In addition to the screen output described above,
\texttt{sem2dsolve} generates different
files and scripts that allow the user
to control the parameters of the simulation and to display the results.
All the outputs files follow the naming convention \texttt{SomeName\_sem2d.xxx},
where \texttt{xxx} is one of the following extensions:
\texttt{tab} for ASCII data files, \texttt{txt} for other text files,
\texttt{dat} for binary data files, etc.
This makes it easy to clean a working directory with a single command like
\texttt{rm -f *\_sem2d*}.

\subsection{Spectral element grid}

As explained in the previous section, 
\texttt{sem2dsolve} generates two PostScript files for 
mesh quality checking purposes:
\texttt{Stability\_sem2d.ps} and \texttt{Resolution\_sem2d.ps}.
The relevant information is contained
in the files \texttt{Stability\_sem2d.tab} and \texttt{Resolution\_sem2d.tab}
and can also be inspected with the Matlab script \texttt{PRE/ViewMeshQuality.m}.

\subsection{Source time function}

\texttt{sem2dsolve} generates a file called
\texttt{SourcesTime\_sem2d.tab} containing
the source time function sampled at the same rate as the receivers.
It is important to verify that the spectrum of the source
has little power at those high frequencies that are not well resolved by the mesh
(those that correspond to less than 5 nodes per wavelength).
If this is not the case you must be very cautious in the interpretation of 
the seismograms in the high frequency range, or low-pass filter the results.

\subsection{Snapshots}

\texttt{sem2dsolve} generates snapshots
at a constant interval defined, in number of solver timesteps, 
by the input parameter \texttt{itd} of the \texttt{SNAP\_DEF} input block.
An example is shown in \figref{snap}.
Requested fields are exported in binary data files called \texttt{xx\_XXX\_sem2d.dat},
where \texttt{xx} is the field code defined in the documentation of the \texttt{PLOTS} input block
and \texttt{XXX} is the 3-digit snapshot number.
The user is encouraged to inspect the Matlab s
function \texttt{POST/sem2d\_snapshot\_read.m}
to find more about the data formats and their manipulation.

Snapshots can also be exported as PostScript files \texttt{xx\_XXX\_sem2d.ps}.
These can be merged into an animated GIF (movie) file \texttt{movie.gif} 
by the script \texttt{POST/movie.csh}
and displayed by \texttt{xanim movie.gif} or \texttt{animate movie.gif}.
An animated GIF can also be created by the 
Matlab function \texttt{POST/sem2d\_snapshot\_movie.m}.

\begin{figure} %[p]
\ImgCL{snapshot.ps}{0.8}
\caption{\label{Fig:snap} Sample snapshot from \texttt{EXAMPLES/UsingEMC2/}:
an obliquely incident SH plane wave impinging on a sedimentary basin. 
The unstructured mesh of spectral elements is plotted on background.}
\end{figure}

\subsection{Seismograms}

The seismograms are stored using the SEP format, a simple
binary block of single precision floats. The components 
of the vector field (velocity by default) are stored
in separate files \texttt{U*\_sem2d.dat},
where \texttt{*} is \texttt{x} or \texttt{z} in P-SV 
and \texttt{y} in SH.
The seismograms header is in the file \texttt{SeisHeader\_sem2d.hdr}.
Its second line contains the sampling timestep \texttt{DT}, the
number of samples \texttt{NSAMP} and the number of stations \texttt{NSTA}.
The stations coordinates, \texttt{XSTA} and \texttt{ZSTA}, are listed
from the third line to the end of file.
With this notations, \texttt{U*\_sem2d.dat} contains a 
\texttt{NSAMP}$\times$\texttt{NSTA} single precision matrix.

You can view the seismograms using any tool
that is able to read the SEP format, which is the case of almost
all the softwares able to deal with seismic data.
\texttt{sem2dsolve} generates scripts for the 
XSU-Seismic Unix visualization tool\footnote{Seismic Unix is freely available from the Colorado 
School of Mines at \url{http://timna.mines.edu/cwpcodes}}:

\begin{sitemize}
\item \texttt{Xline\_sem2d.csh} displays all seismograms together on screen
\item \texttt{PSline\_sem2d.csh} plots all seismograms
on PostScript files \texttt{U*Poly\_sem2d.ps}
\item \texttt{Xtrace\_sem2d.csh} prompts the user for a trace number
(between 1 and \texttt{NSTA})
and then displays this particular trace on screen
\item \texttt{PStrace\_sem2d.csh} does the same as \texttt{Xtrace},
but exports the traces as PostScript files \texttt{U*TraceXXX\_sem2d.ps} 
where \texttt{XXX} is the number of that particular trace
\end{sitemize}

The program \texttt{post\_seis.exe} performs similar
basic manipulation and plotting (through \texttt{gnuplot}) of the 
seismograms. Its interactive menu is self-explanatory.
It is usually called inside a script, as in \texttt{POST/seis\_b2a.csh}
(converts all seismograms to ASCII) or \texttt{POST/seis\_plot.csh}
(plots all seismograms together, an example is shown in \figref{seis}).

The script \texttt{POST/sample\_seis.m} shows how
to manipulate and plot seismogram data in Matlab.
It uses the functions \texttt{POST/sem2d\_read\_seis.m} and \texttt{POST/plot\_seis.m}.

\begin{figure} %[p]
\ImgCL{Uy_sem2d.ps}{0.58}
\caption{\label{Fig:seis} Sample seismograms from \texttt{EXAMPLES/UsingEMC2/}
generated with \texttt{POST/seis\_plot.csh}.}
\end{figure}

\subsection{Fault outputs}

Fault data from dynamic rupture simulations is stored
in three files (where \texttt{XX} is the boundary tag of the first
side of the fault, \texttt{tags(1)} of the \texttt{BC\_SWFFLT} input block):
\begin{sitemize}
  \item \texttt{FltXX\_sem2d.hdr} contains the information needed
to read the other fault data files. Its format, line by line, is:
\begin{enumerate}
\item \texttt{NPTS NDAT NSAMP DELT} (name of parameters) 
\item Value of parameters above
\item Name of fields exported in \texttt{FltXX\_sem2d.dat}, separated by ``:"
\item \texttt{XPTS ZPTS} (name of coordinate axis)
\item from here to the end of file: a two-column table of coordinates
of the output fault nodes
\end{enumerate}

  \item \texttt{FltXX\_sem2d.dat} contains the space-time distribution of fault data such as slip, slip rate, stress and strength. 
Every \texttt{DELT} seconds a block of fault data values is written. The total number of blocks is \texttt{NSAMP}.
Each block has \texttt{NDAT} lines (one per fault data field) 
and \texttt{NPTS} columns (one per fault node)
\footnote{The actual number of columns is \texttt{NPTS} +2:
Fortran adds a one-word tag at the front and end of each record.}. 
Stresses are relative to their initial values.

  \item \texttt{FltXX\_init\_sem2d.tab} contains the spatial distribution of 
initial shear stress, initial normal stress and initial friction (3 columns).

  \item \texttt{FltXX\_potency\_sem2d.tab} contains time-series of seismic potency and potency rate. 
The seismic potency tensor $p_{ij}$ is defined by the following integral along the fault:
\begin{equation}
p_{ij} = \frac{1}{2} \int_{fault} (n_i \Delta u_j + n_j \Delta u_i)  \ dx
\end{equation}
where $\Delta u$ is slip and $n$ is the local unit vector normal to the fault.
The file contains one line per timestep.
In SH (\texttt{ndof=1}) each line has $4$ columns: $2$ components of potency ($p_{13}$ and $p_{23}$)
and $2$ components of potency rate ($\dot{p}_{13}$ and $\dot{p}_{23}$).
In P-SV (\texttt{ndof=2}) each line has $3$ components of potency ($p_{11}$, $p_{22}$ and $p_{12}$)
and $3$ components of potency rate ($\dot{p}_{11}$, $\dot{p}_{22}$ and $\dot{p}_{12}$).
\end{sitemize}

Some tools are available to manipulate the data in \texttt{FltXX\_sem2d.dat}:
\begin{sitemize}
\item The script \texttt{FltXX\_sem2d.csh} shows 
how to extract ASCII time series of different 
fields at given locations on the fault, using Seismic Unix tools.
\item The program \texttt{post\_fault.exe} performs basic manipulations of the 
fault data, including conversion to an ASCII file readable by \texttt{gnuplot}. 
Its interactive menu is self-explanatory.
\item The script \texttt{POST/sample\_fault.m} 
and function \texttt{POST/sem2d\_read\_fault.m} show how
to manipulate and plot fault data in Matlab.
\end{sitemize}

\subsection{Stress glut}

For damage and plastic materials,
the solver can export the plastic and damage components of the
cumulative stress glut tensors 
defined, respectively, as
\eqa
  s^p_{ij}(t) &=& -\iint 2\mu\, \epsilon^{p}_{ij}(t) \ dx\,dz\\
  s^d_{ij}(t) &=& \iint [ \sigma_{ij}(t) - c^0_{ijkl} \epsilon^e_{kl}(t) ]  \ dx\,dz
\ena
where $\epsilon^p$ is the plastic strain, $\epsilon^e$ the elastic strain,
$\sigma$ the absolute stress
and $c^0$ the tensor of elastic moduli of the undamaged medium. 

To enable this feature: set \texttt{COMPUTE\_STRESS\_GLUT = .true.} 
in file \texttt{SRC/constants.f90}, then re-compile the code.
The stress glut output is exported in the file \texttt{stress\_glut\_sem2d.tab}
in 7 columns: time, $s^p_{11}$, $s^p_{22}$, $s^p_{12}$, $s^d_{11}$, $s^d_{22}$, $s^d_{12}$.

\subsection{Energies}

The solver can export the 
cumulative plastic energy, the kinetic energy and the total change of elastic energy,
defined respectively by
\eqa
 E^p(t) &=& \iiint_0^t \sigma_{ij}(t') \dot{\epsilon}^p_{ij}(t') \ dx\,dz\,dt'\\
 E^k(t) &=& 1/2\ \iint \rho v_i^2(t) \ dx\,dz\\
 E^e(t) &=& \iint U[\epsilon^e(t)] - U[\epsilon^e(0)] \ dx\,dz
\ena
where $U$ is the elastic potential and summation over subindices is implied.

To enable this feature: set \texttt{COMPUTE\_ENERGIES = .true.}
in file \texttt{SRC/constants.f90}, then re-compile the code.
The energy output is exported in the file \texttt{energy\_sem2d.tab} 
in 4 columns: time, $E^p$, $E^k$ and $E_e$.

\subsection{Matlab utilities}
\label{Sec:matlab}

A range of functions and sample scripts for Matlab are available 
to read, manipulate and plot output data.
Add the directory \texttt{POST/} to your Matlab path (\texttt{addpath}).
For an overview of existing utilities, type \texttt{help POST}:

\begin{verbatim}
   SEM2DPACK/POST provides Matlab utilities for the manipulation
   and visualization of SEM2DPACK simulations results.

   Reading simulation data:

    SEM2D_READ_SPECGRID reads a spectral element grid
    SEM2D_SNAPSHOT_READ reads snapshot data
    SEM2D_READ_SEIS     reads seismogram data
    SEM2D_READ_FAULT    reads fault data

   Data manipulation:

    SEM2D_EXTRACT_POINT extracts field values at an arbitrary point
    SEM2D_EXTRACT_LINE  extracts field values along a vertical or horizontal line
    ARIAS_INTENSITY     computes Arias Intensity and Significant Duration
    RESPONSE_SPECTRUM   computes response spectra (peak dynamic response
                        of single-degree-of-freedom systems)

   Data visualization:

    SEM2D_PLOT_GRID     plots a spectral element grid
    SEM2D_SNAPSHOT_PLOT plots snapshot data
    SEM2D_SNAPSHOT_GUI  interactively plots snapshot data
    SEM2D_SNAPSHOT_MOVIE makes an animation of snapshot data
    PLOT_MODEL          plots velocity and density model
    PLOT_SEIS           plots multiple seismograms
    PLOT_FRONTS         space-time plot of rupture front and process zone tail
    SAMPLE_FAULT        example of visualization of fault data
    SAMPLE_SEIS         example of visualization of seismogram data

   Miscellaneous tools:

    XCORRSHIFT          cross-correlation time-delay measurement
    SPECSHIFT           signal time shift by non-integer lag via spectral domain
    SPECFILTER          zero-phase Butterworth filter via spectral domain

\end{verbatim}
